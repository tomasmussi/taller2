% Generated by Sphinx.
\def\sphinxdocclass{report}
\newif\ifsphinxKeepOldNames \sphinxKeepOldNamestrue
\documentclass[a4paper,10pt,spanish]{sphinxmanual}
\usepackage{iftex}

\ifPDFTeX
  \usepackage[utf8]{inputenc}
\fi
\ifdefined\DeclareUnicodeCharacter
  \DeclareUnicodeCharacter{00A0}{\nobreakspace}
\fi
\usepackage{cmap}
\usepackage[T1]{fontenc}
\usepackage{amsmath,amssymb,amstext}
\usepackage{babel}
\usepackage{times}
\usepackage[Sonny]{fncychap}
\usepackage{longtable}
\usepackage{sphinx}
\usepackage{multirow}
\usepackage{eqparbox}


\addto\captionsspanish{\renewcommand{\figurename}{Figura }}
\addto\captionsspanish{\renewcommand{\tablename}{Tabla }}
\SetupFloatingEnvironment{literal-block}{name=Lista }

\addto\extrasspanish{\def\pageautorefname{página}}

\setcounter{tocdepth}{2}


\title{Linkedun Documentación Técnica}
\date{dic. 01, 2016}
\release{1}
\author{Mussi, Tomás \\ Arancibia, Luis \\ Dufau, Ezequiel \\ Ackermann, Alfredo}
\newcommand{\sphinxlogo}{}
\renewcommand{\releasename}{Manual del desarrollador}
\makeindex

\makeatletter
\def\PYG@reset{\let\PYG@it=\relax \let\PYG@bf=\relax%
    \let\PYG@ul=\relax \let\PYG@tc=\relax%
    \let\PYG@bc=\relax \let\PYG@ff=\relax}
\def\PYG@tok#1{\csname PYG@tok@#1\endcsname}
\def\PYG@toks#1+{\ifx\relax#1\empty\else%
    \PYG@tok{#1}\expandafter\PYG@toks\fi}
\def\PYG@do#1{\PYG@bc{\PYG@tc{\PYG@ul{%
    \PYG@it{\PYG@bf{\PYG@ff{#1}}}}}}}
\def\PYG#1#2{\PYG@reset\PYG@toks#1+\relax+\PYG@do{#2}}

\expandafter\def\csname PYG@tok@gd\endcsname{\def\PYG@tc##1{\textcolor[rgb]{0.63,0.00,0.00}{##1}}}
\expandafter\def\csname PYG@tok@gu\endcsname{\let\PYG@bf=\textbf\def\PYG@tc##1{\textcolor[rgb]{0.50,0.00,0.50}{##1}}}
\expandafter\def\csname PYG@tok@gt\endcsname{\def\PYG@tc##1{\textcolor[rgb]{0.00,0.27,0.87}{##1}}}
\expandafter\def\csname PYG@tok@gs\endcsname{\let\PYG@bf=\textbf}
\expandafter\def\csname PYG@tok@gr\endcsname{\def\PYG@tc##1{\textcolor[rgb]{1.00,0.00,0.00}{##1}}}
\expandafter\def\csname PYG@tok@cm\endcsname{\let\PYG@it=\textit\def\PYG@tc##1{\textcolor[rgb]{0.25,0.50,0.56}{##1}}}
\expandafter\def\csname PYG@tok@vg\endcsname{\def\PYG@tc##1{\textcolor[rgb]{0.73,0.38,0.84}{##1}}}
\expandafter\def\csname PYG@tok@vi\endcsname{\def\PYG@tc##1{\textcolor[rgb]{0.73,0.38,0.84}{##1}}}
\expandafter\def\csname PYG@tok@mh\endcsname{\def\PYG@tc##1{\textcolor[rgb]{0.13,0.50,0.31}{##1}}}
\expandafter\def\csname PYG@tok@cs\endcsname{\def\PYG@tc##1{\textcolor[rgb]{0.25,0.50,0.56}{##1}}\def\PYG@bc##1{\setlength{\fboxsep}{0pt}\colorbox[rgb]{1.00,0.94,0.94}{\strut ##1}}}
\expandafter\def\csname PYG@tok@ge\endcsname{\let\PYG@it=\textit}
\expandafter\def\csname PYG@tok@vc\endcsname{\def\PYG@tc##1{\textcolor[rgb]{0.73,0.38,0.84}{##1}}}
\expandafter\def\csname PYG@tok@il\endcsname{\def\PYG@tc##1{\textcolor[rgb]{0.13,0.50,0.31}{##1}}}
\expandafter\def\csname PYG@tok@go\endcsname{\def\PYG@tc##1{\textcolor[rgb]{0.20,0.20,0.20}{##1}}}
\expandafter\def\csname PYG@tok@cp\endcsname{\def\PYG@tc##1{\textcolor[rgb]{0.00,0.44,0.13}{##1}}}
\expandafter\def\csname PYG@tok@gi\endcsname{\def\PYG@tc##1{\textcolor[rgb]{0.00,0.63,0.00}{##1}}}
\expandafter\def\csname PYG@tok@gh\endcsname{\let\PYG@bf=\textbf\def\PYG@tc##1{\textcolor[rgb]{0.00,0.00,0.50}{##1}}}
\expandafter\def\csname PYG@tok@ni\endcsname{\let\PYG@bf=\textbf\def\PYG@tc##1{\textcolor[rgb]{0.84,0.33,0.22}{##1}}}
\expandafter\def\csname PYG@tok@nl\endcsname{\let\PYG@bf=\textbf\def\PYG@tc##1{\textcolor[rgb]{0.00,0.13,0.44}{##1}}}
\expandafter\def\csname PYG@tok@nn\endcsname{\let\PYG@bf=\textbf\def\PYG@tc##1{\textcolor[rgb]{0.05,0.52,0.71}{##1}}}
\expandafter\def\csname PYG@tok@no\endcsname{\def\PYG@tc##1{\textcolor[rgb]{0.38,0.68,0.84}{##1}}}
\expandafter\def\csname PYG@tok@na\endcsname{\def\PYG@tc##1{\textcolor[rgb]{0.25,0.44,0.63}{##1}}}
\expandafter\def\csname PYG@tok@nb\endcsname{\def\PYG@tc##1{\textcolor[rgb]{0.00,0.44,0.13}{##1}}}
\expandafter\def\csname PYG@tok@nc\endcsname{\let\PYG@bf=\textbf\def\PYG@tc##1{\textcolor[rgb]{0.05,0.52,0.71}{##1}}}
\expandafter\def\csname PYG@tok@nd\endcsname{\let\PYG@bf=\textbf\def\PYG@tc##1{\textcolor[rgb]{0.33,0.33,0.33}{##1}}}
\expandafter\def\csname PYG@tok@ne\endcsname{\def\PYG@tc##1{\textcolor[rgb]{0.00,0.44,0.13}{##1}}}
\expandafter\def\csname PYG@tok@nf\endcsname{\def\PYG@tc##1{\textcolor[rgb]{0.02,0.16,0.49}{##1}}}
\expandafter\def\csname PYG@tok@si\endcsname{\let\PYG@it=\textit\def\PYG@tc##1{\textcolor[rgb]{0.44,0.63,0.82}{##1}}}
\expandafter\def\csname PYG@tok@s2\endcsname{\def\PYG@tc##1{\textcolor[rgb]{0.25,0.44,0.63}{##1}}}
\expandafter\def\csname PYG@tok@nt\endcsname{\let\PYG@bf=\textbf\def\PYG@tc##1{\textcolor[rgb]{0.02,0.16,0.45}{##1}}}
\expandafter\def\csname PYG@tok@nv\endcsname{\def\PYG@tc##1{\textcolor[rgb]{0.73,0.38,0.84}{##1}}}
\expandafter\def\csname PYG@tok@s1\endcsname{\def\PYG@tc##1{\textcolor[rgb]{0.25,0.44,0.63}{##1}}}
\expandafter\def\csname PYG@tok@ch\endcsname{\let\PYG@it=\textit\def\PYG@tc##1{\textcolor[rgb]{0.25,0.50,0.56}{##1}}}
\expandafter\def\csname PYG@tok@m\endcsname{\def\PYG@tc##1{\textcolor[rgb]{0.13,0.50,0.31}{##1}}}
\expandafter\def\csname PYG@tok@gp\endcsname{\let\PYG@bf=\textbf\def\PYG@tc##1{\textcolor[rgb]{0.78,0.36,0.04}{##1}}}
\expandafter\def\csname PYG@tok@sh\endcsname{\def\PYG@tc##1{\textcolor[rgb]{0.25,0.44,0.63}{##1}}}
\expandafter\def\csname PYG@tok@ow\endcsname{\let\PYG@bf=\textbf\def\PYG@tc##1{\textcolor[rgb]{0.00,0.44,0.13}{##1}}}
\expandafter\def\csname PYG@tok@sx\endcsname{\def\PYG@tc##1{\textcolor[rgb]{0.78,0.36,0.04}{##1}}}
\expandafter\def\csname PYG@tok@bp\endcsname{\def\PYG@tc##1{\textcolor[rgb]{0.00,0.44,0.13}{##1}}}
\expandafter\def\csname PYG@tok@c1\endcsname{\let\PYG@it=\textit\def\PYG@tc##1{\textcolor[rgb]{0.25,0.50,0.56}{##1}}}
\expandafter\def\csname PYG@tok@o\endcsname{\def\PYG@tc##1{\textcolor[rgb]{0.40,0.40,0.40}{##1}}}
\expandafter\def\csname PYG@tok@kc\endcsname{\let\PYG@bf=\textbf\def\PYG@tc##1{\textcolor[rgb]{0.00,0.44,0.13}{##1}}}
\expandafter\def\csname PYG@tok@c\endcsname{\let\PYG@it=\textit\def\PYG@tc##1{\textcolor[rgb]{0.25,0.50,0.56}{##1}}}
\expandafter\def\csname PYG@tok@mf\endcsname{\def\PYG@tc##1{\textcolor[rgb]{0.13,0.50,0.31}{##1}}}
\expandafter\def\csname PYG@tok@err\endcsname{\def\PYG@bc##1{\setlength{\fboxsep}{0pt}\fcolorbox[rgb]{1.00,0.00,0.00}{1,1,1}{\strut ##1}}}
\expandafter\def\csname PYG@tok@mb\endcsname{\def\PYG@tc##1{\textcolor[rgb]{0.13,0.50,0.31}{##1}}}
\expandafter\def\csname PYG@tok@ss\endcsname{\def\PYG@tc##1{\textcolor[rgb]{0.32,0.47,0.09}{##1}}}
\expandafter\def\csname PYG@tok@sr\endcsname{\def\PYG@tc##1{\textcolor[rgb]{0.14,0.33,0.53}{##1}}}
\expandafter\def\csname PYG@tok@mo\endcsname{\def\PYG@tc##1{\textcolor[rgb]{0.13,0.50,0.31}{##1}}}
\expandafter\def\csname PYG@tok@kd\endcsname{\let\PYG@bf=\textbf\def\PYG@tc##1{\textcolor[rgb]{0.00,0.44,0.13}{##1}}}
\expandafter\def\csname PYG@tok@mi\endcsname{\def\PYG@tc##1{\textcolor[rgb]{0.13,0.50,0.31}{##1}}}
\expandafter\def\csname PYG@tok@kn\endcsname{\let\PYG@bf=\textbf\def\PYG@tc##1{\textcolor[rgb]{0.00,0.44,0.13}{##1}}}
\expandafter\def\csname PYG@tok@cpf\endcsname{\let\PYG@it=\textit\def\PYG@tc##1{\textcolor[rgb]{0.25,0.50,0.56}{##1}}}
\expandafter\def\csname PYG@tok@kr\endcsname{\let\PYG@bf=\textbf\def\PYG@tc##1{\textcolor[rgb]{0.00,0.44,0.13}{##1}}}
\expandafter\def\csname PYG@tok@s\endcsname{\def\PYG@tc##1{\textcolor[rgb]{0.25,0.44,0.63}{##1}}}
\expandafter\def\csname PYG@tok@kp\endcsname{\def\PYG@tc##1{\textcolor[rgb]{0.00,0.44,0.13}{##1}}}
\expandafter\def\csname PYG@tok@w\endcsname{\def\PYG@tc##1{\textcolor[rgb]{0.73,0.73,0.73}{##1}}}
\expandafter\def\csname PYG@tok@kt\endcsname{\def\PYG@tc##1{\textcolor[rgb]{0.56,0.13,0.00}{##1}}}
\expandafter\def\csname PYG@tok@sc\endcsname{\def\PYG@tc##1{\textcolor[rgb]{0.25,0.44,0.63}{##1}}}
\expandafter\def\csname PYG@tok@sb\endcsname{\def\PYG@tc##1{\textcolor[rgb]{0.25,0.44,0.63}{##1}}}
\expandafter\def\csname PYG@tok@k\endcsname{\let\PYG@bf=\textbf\def\PYG@tc##1{\textcolor[rgb]{0.00,0.44,0.13}{##1}}}
\expandafter\def\csname PYG@tok@se\endcsname{\let\PYG@bf=\textbf\def\PYG@tc##1{\textcolor[rgb]{0.25,0.44,0.63}{##1}}}
\expandafter\def\csname PYG@tok@sd\endcsname{\let\PYG@it=\textit\def\PYG@tc##1{\textcolor[rgb]{0.25,0.44,0.63}{##1}}}

\def\PYGZbs{\char`\\}
\def\PYGZus{\char`\_}
\def\PYGZob{\char`\{}
\def\PYGZcb{\char`\}}
\def\PYGZca{\char`\^}
\def\PYGZam{\char`\&}
\def\PYGZlt{\char`\<}
\def\PYGZgt{\char`\>}
\def\PYGZsh{\char`\#}
\def\PYGZpc{\char`\%}
\def\PYGZdl{\char`\$}
\def\PYGZhy{\char`\-}
\def\PYGZsq{\char`\'}
\def\PYGZdq{\char`\"}
\def\PYGZti{\char`\~}
% for compatibility with earlier versions
\def\PYGZat{@}
\def\PYGZlb{[}
\def\PYGZrb{]}
\makeatother

\renewcommand\PYGZsq{\textquotesingle}

\begin{document}
\shorthandoff{"}
\maketitle
\tableofcontents
\phantomsection\label{index::doc}



\chapter{AppServer}
\label{index:linkedun-manual-tecnico}\label{index:appserver}

\section{Documentación técnica del AppServer}
\label{appServer::doc}\label{appServer:documentacion-tecnica-del-appserver}

\subsection{Objetivo}
\label{docTecnica:objetivo}\label{docTecnica::doc}
Documentar información técnica del app server para poder compilar, ejecutar, debuggear, extender y mejorar el Application Server.


\subsection{Requisitos}
\label{docTecnica:requisitos}
El servidor de la aplicación Linkedun al cuál se consultan todos los servicios está desarrollado en C++, standard de 2011 (C++11) y utiliza las siguientes bibliotecas:
\begin{itemize}
\item {} 
Mongoose cpp{[}1{]} como web server para responder todas las consultas realizadas por la aplicación Android.

\item {} 
Jsoncpp {[}2{]} biblioteca de JSON para C++.

\item {} 
Curlpp {[}3{]} versión 0.7.3 o superior.

\item {} 
GoogleTest {[}4{]} con tests unitarios que verifican que no se haya modificado funcionalidad existente. No es estrictamente necesario para continuar con el desarrollo de la aplicación, pero es ampliamente recomendado realizar tests y continuar con el desarrollo TDD {[}5{]}.

\item {} 
Leveldb {[}6{]} la base de datos clave-valor utilizada para guardar los datos de usuarios y la interacción entre los mismos.

\item {} 
Log4Cpp {[}7{]} biblioteca de logging para C++.

\item {} 
lcov \& gcov {[}8{]} aplicaciones para verificar el code coverage que tienen los tests sobre la aplicación.

\item {} 
CMake {[}9{]} para compilar el proyecto entero con todas las dependencias y también compilar y ejecutar los tests unitarios.

\end{itemize}

El código distribuido únicamente compila estáticamente la biblioteca de mongoose para asegurar que no existan problemas de compatibilidad al instalar, pero el resto de las bibliotecas se tienen que instalar y linkear dinámicamente con el ejecutable.
Se utiliza bibliotecas de terceros en todo el código, especialmente las bibliotecas de mongoose y jsoncpp. En el caso de la base de datos y la ejecución de peticiones http al otro servidor mediante curlpp se hizo una interfaz para lograr separar la utilización de la biblioteca y poder reemplazar dicha interfaz en caso de querer cambiar alguna biblioteca de terceros, es decir, usar alguna otra biblioteca de peticiones http e invocar directamente un comando del sistema operativo por ejemplo.


\subsection{Distribución}
\label{docTecnica:distribucion}
El código puede ser descargado de la página oficial de github {[}10{]}.
Sistema operativo
El desarrollo de la aplicación se ha realizado bajo el sistema operativo Linux. De querer desarrollar en Windows o Mac, deberá considerar la portabilidad de todas las dependencias a esos sistemas operativos y cómo realizar la compilación del proyecto.


\subsection{Compilación}
\label{docTecnica:compilacion}
Para compilar el app server, puede seguir cualquiera de los dos métodos:
1. Utilizar la imágen docker y ejecutar el comando docker run que iniciará una imagen con todas las dependencias instaladas y tendrá listo el servidor ejecutándose{[}a{]}, como se detalla en el manual de instalación de servidores {[}11{]}.
2. Instalar todas las dependencias mencionadas y realizar los siguientes pasos:
\begin{enumerate}
\item {} 
Ir al directorio “server”

\item {} 
Invocar el script: “. ./appear.sh” que se encarga de invocar a cmake con las opciones de compilación predeterminadas.

\end{enumerate}


\subsection{Ejecución}
\label{docTecnica:ejecucion}
Debería estar en el directorio de la carpeta build y de ahí puede ejecutar el binario y con esa invocación el servidor estará escuchando conexiones en el puerto 8080. Al binario del servidor se le pueden pasar dos parámetros opcionales, en el orden mencionado, para cambiar el puerto donde el servidor recibe las peticiones y la url del shared server. Ejemplos de ejecución:
\begin{itemize}
\item {} 
./src/server

\item {} 
./src/server 7000

\item {} 
./src/server 8080 localhost:5642

\item {} 
./src/server 7000 \url{https://guarded-sands-84788.herokuapp.com}

\item {} 
./src/server 8080 \url{https://guarded-sands-84788.herokuapp.com} (esta es la invocación por defecto)

\end{itemize}

Si se define el puerto donde se escucharán conexiones, se debe especificar si o si la url del shared server. Cualquier otra invocación es incorrecta y derivará en errores en la aplicación.
Estructura
La estructura de la aplicación es un Controller de Mongoose que envía respuestas en formato JSON (para más detalles ver la API especificada), la clase que se encarga de todas las solicitudes es ApiJsonController y deriva las consultas en los handlers correspondientes. Un ejemplo con un diagrama de secuencia clarificará este párrafo.

\noindent\sphinxincludegraphics{{getContacts}.png}

Los handlers, los cuáles son consultados por el ApiJsonController son:
\begin{itemize}
\item {} 
UserHandler: se encarga de la interacción entre los usuarios, las peticiones de contactar, votar, responder consultas y otros servicios.

\item {} 
UserList: se encarga de mantener un listado actualizado de los usuarios, tanto darlos de alta en un registro, como consultar para ver qué usuarios están disponibles en el sistema.

\item {} 
GoogleCloudMessagesHandler: se utiliza para enviar notificaciones cuando un usuario le envía un mensaje a otro.

\item {} 
DatabaseHandler: se encarga de mantener realizar las operaciones de lectura y escritura en la base de datos. Simplemente tiene 3 métodos los cuales permiten realizar las operaciones necesarias.

\end{itemize}

Las entidades a destacar del sistema son:
\begin{itemize}
\item {} 
User: representa a los usuarios y es la clase fundamental dado que es una aplicación de interacción entre usuarios.

\item {} 
Chat: representa un chat entre dos personas. El mismo tiene una lista de mensajes para mostrar que fueron enviando ambos y se puede reconstruir la conversación entera.

\item {} 
Message: representa un mensaje enviado por un usuario. Se utiliza un timestamp guardado como string de forma de identificar el orden de los mensajes dentro del chat.

\item {} 
HerokuService: deriva todas las consultas que deben ser realizadas al otro servidor y devuelve las respuestas a la aplicación de android tal cual son recibidas.

\item {} 
Log: loguea toda la actividad del sistema en un archivo. Es una interfaz simple que utiliza Log4cpp para la finalidad de loguear.

\end{itemize}

Base de datos
Debido a un requisito no funcional, se utilizó una base de datos clave-valor.
Los usuarios, su información y las interacciones se persisten en la base de datos con una clave identificatoria según si es usuario o con un identificador especial para referirse a alguna entidad como la lista de usuarios. Las entidades de dominio están todas serializadas como JSON y se almacenan de esa manera en la base de datos.
ACLARACIÓN: cuando se encuentra algo entre los caracteres “\textless{}\textgreater{}” significa que se reemplaza la cadena dentro por el valor de la entidad. Por ejemplo, si tengo user-\textless{}user-id\textgreater{} una clave de un usuario en particular será user-johndoeid.

\begin{Verbatim}[commandchars=\\\{\}]
Entidad
        Clave
        Valor
        User
        user\PYGZhy{}\PYGZlt{}user\PYGZhy{}id\PYGZgt{}
        \PYGZob{}
              \PYGZdq{}user\PYGZdq{} :
              \PYGZob{}
                      \PYGZdq{}city\PYGZdq{} : \PYGZdq{}CABA\PYGZdq{},
                      \PYGZdq{}dob\PYGZdq{} : \PYGZdq{}11/07/1991\PYGZdq{},
                      \PYGZdq{}email\PYGZdq{} : \PYGZdq{}tomas@gmail.com\PYGZdq{},
                      \PYGZdq{}fb\PYGZus{}id\PYGZdq{} : \PYGZdq{}\PYGZdq{},
                      \PYGZdq{}friends\PYGZdq{} : [],
                      \PYGZdq{}job\PYGZus{}positions\PYGZdq{} : [],
                      \PYGZdq{}latitude\PYGZdq{} : \PYGZdq{}\PYGZdq{},
                      \PYGZdq{}longitude\PYGZdq{} : \PYGZdq{}\PYGZdq{},
                      \PYGZdq{}name\PYGZdq{} : \PYGZdq{}Tomas Mussi\PYGZdq{},
                      \PYGZdq{}profile\PYGZus{}photo\PYGZdq{} : \PYGZdq{}QdHVuZw==\PYGZdq{},
                      \PYGZdq{}requests\PYGZdq{} : [],
                      \PYGZdq{}skills\PYGZdq{} : [],
                      \PYGZdq{}summary\PYGZdq{} : \PYGZdq{}Estudiante UBA.\PYGZdq{},
                      \PYGZdq{}votes\PYGZdq{} : []
              \PYGZcb{}
\PYGZcb{}
        UserList
        users
        \PYGZob{}
         \PYGZdq{}users\PYGZdq{}:[
            \PYGZdq{}fb\PYGZus{}id\PYGZus{}tomas\PYGZdq{},
            \PYGZdq{}fb\PYGZus{}id\PYGZus{}luis\PYGZdq{}
         ]
\PYGZcb{}


        Chat
        chat\PYGZhy{}\PYGZlt{}userid\PYGZhy{}1\PYGZgt{}\PYGZhy{}\PYGZlt{}userid2\PYGZgt{}
        \PYGZob{}
\PYGZdq{}messages\PYGZdq{} :
         [
            \PYGZob{}
              \PYGZdq{}message\PYGZdq{} : \PYGZdq{}como va?\PYGZdq{},
              \PYGZdq{}receiver\PYGZus{}id\PYGZdq{} : \PYGZdq{}recv\PYGZdq{},
              \PYGZdq{}sender\PYGZus{}id\PYGZdq{} : \PYGZdq{}send\PYGZdq{},
              \PYGZdq{}timestamp\PYGZdq{} : \PYGZdq{}2016\PYGZhy{}11\PYGZhy{}14 15:28:43\PYGZdq{}
            \PYGZcb{},
            \PYGZob{}
              \PYGZdq{}message\PYGZdq{} : \PYGZdq{}hola\PYGZdq{},
              \PYGZdq{}receiver\PYGZus{}id\PYGZdq{} : \PYGZdq{}recv\PYGZdq{},
              \PYGZdq{}sender\PYGZus{}id\PYGZdq{} : \PYGZdq{}send\PYGZdq{},
              \PYGZdq{}timestamp\PYGZdq{} : \PYGZdq{}2016\PYGZhy{}11\PYGZhy{}14 13:57:23\PYGZdq{}
            \PYGZcb{}
         ]
\PYGZcb{}
\end{Verbatim}

Respecto a un chat en especial, el \textless{}userid-1\textgreater{} será el del iniciador de la conversación, si uno quiere recuperar esa conversación y se desconoce quién inició la conversación, deberá probarse las dos combinaciones de ids para poder saber si existía una conversación previa entre los usuarios o hay que crear un chat nuevo. Esta resolución es engorrosa y en la próxima versión se contempla realizar un ordenamiento alfabético de los ids de usuario para no tener que probar las permutaciones de ids.
API
El app server respeta la API-REST{[}12{]} de interacción entre la aplicación Android y el servidor.


\subsection{Contacto}
\label{docTecnica:contacto}
Por cualquier inquietud, duda, consulta, usted puede enviar un email a la casilla de correos \href{mailto:linkedunservices@gmail.com}{linkedunservices@gmail.com} y se le responderá a la brevedad.
También puede contribuir al proyecto, reportar bugs, etc en la página de github del proyecto.


\subsection{Referencias}
\label{docTecnica:referencias}\begin{itemize}
\item {} 
{[}1{]} Mongoose-cpp: \url{https://github.com/Gregwar/mongoose-cpp}

\item {} 
{[}2{]} Jsoncpp: \url{https://github.com/open-source-parsers/jsoncpp}

\item {} 
{[}3{]} Curlpp: \url{http://www.curlpp.org/}

\item {} 
{[}4{]} Google Test: \url{https://github.com/google/googletest}

\item {} 
{[}5{]} Test Driven Development: \url{https://en.wikipedia.org/wiki/Test-driven\_development}

\item {} 
{[}6{]} Level DB: \url{https://github.com/google/leveldb}

\item {} 
{[}7{]} Log4Cpp: \url{http://log4cpp.sourceforge.net/}

\item {} 
{[}8{]} Gcov: \url{https://gcc.gnu.org/onlinedocs/gcc/Gcov.html}

\item {} 
{[}9{]} Cmake: \url{https://cmake.org/}

\item {} 
{[}10{]} Código del repositorio oficial de Linkedun: \url{https://github.com/tomasmussi/taller2}

\item {} 
{[}11{]} Manual de instalación de servidores: \url{https://docs.google.com/document/d/1kZQ\_5mHo0CJr2s9m6N1SgbrNDUm87tNNwNeo0STK0o/edit\#heading=h.vo1cx285zg1d}

\end{itemize}


\section{Especificación Técnica APIRest}
\label{appServer:especificacion-tecnica-apirest}

\subsection{Objetivo}
\label{apiRest:objetivo}\label{apiRest::doc}
Definir el protocolo de comunicación entre la aplicación android y el app server que tenemos que implementar.
Funcionalidad: recordar que la funcionalidad pedida en el trabajo es:
\begin{itemize}
\item {} 
Loguearme en el sistema

\item {} 
Ver/modificar mi perfil

\item {} 
Buscar a otro usuario  (1) (lookup)

\item {} 
Invitar a conectar a otro usuario (add\_contact)

\item {} 
Aceptar/cancelar una solicitud de conectar (accept\_contact)

\item {} 
Get my contacts…(get\_contacts)

\item {} 
Recomendar a un usuario (votar por un usuario)

\item {} 
Consultar usuarios más populares

\item {} 
Chatear con usuarios con los que haya conectado

\item {} 
Ver/modificar Skills

\item {} 
Ver/modificar job positions

\item {} 
Cambiar mi ubicación actual

\end{itemize}


\subsection{Servicio de autenticación}
\label{apiRest:servicio-de-autenticacion}
\begin{Verbatim}[commandchars=\\\{\}]
LoginDTO
URL
        URL: http://\PYGZlt{}appserver\PYGZgt{}/api/fb\PYGZus{}login
        HTTP VERB
        GET
        PARAMETROS
        \PYGZob{}
              “api\PYGZus{}sec” : “api\PYGZus{}sec\PYGZus{}hash”,
              “fb\PYGZus{}user\PYGZus{}id”: “a\PYGZus{}fb\PYGZus{}user\PYGZus{}id”
\PYGZcb{}
        RESPUESTA
        \PYGZob{}
        \PYGZdq{}data\PYGZdq{}:
           \PYGZob{}
              \PYGZdq{}token\PYGZdq{}:\PYGZdq{}d975c4e93772b4fcbdf65ce62a1e14f6\PYGZdq{},
              \PYGZdq{}user\PYGZus{}exists\PYGZdq{}:\PYGZdq{}false\PYGZdq{}
           \PYGZcb{}
        ,
        \PYGZdq{}errors\PYGZdq{}:[
        ]
\PYGZcb{}
        RESPUESTA
        \PYGZob{}
         “data”: null,
         “errors”: [
              “status” : “ERROR”,
              “message” : “Facebook user ID invalido”
         ]
\PYGZcb{}
\end{Verbatim}


\subsection{Servicio de búsqueda de profesionales}
\label{apiRest:servicio-de-busqueda-de-profesionales}
\begin{Verbatim}[commandchars=\\\{\}]
URL
        URL: http://\PYGZlt{}appserver\PYGZgt{}/api/lookup
        HTTP VERB
        GET
        PARAMETROS
        \PYGZob{}
              \PYGZdq{}token\PYGZdq{}: “hash”,
              “query” : “tomas”
\PYGZcb{}
        RESPUESTA
        \PYGZob{}
         \PYGZdq{}data\PYGZdq{}:[
            [
               \PYGZob{}
                  \PYGZdq{}distance”:”3.978116”,
                  \PYGZdq{}fb\PYGZus{}id\PYGZdq{}:\PYGZdq{}tomas\PYGZdq{},
                  \PYGZdq{}is\PYGZus{}contact\PYGZdq{}:\PYGZdq{}false\PYGZdq{},
                  \PYGZdq{}name\PYGZdq{}:\PYGZdq{}Tomas mussi\PYGZdq{},
                  \PYGZdq{}photo\PYGZdq{}:\PYGZdq{}bas64encode..==\PYGZdq{}
               \PYGZcb{}
            ]
         ],
         \PYGZdq{}errors\PYGZdq{}:[


         ]
\PYGZcb{}
\end{Verbatim}


\subsection{Servicio de actualización de ubicación}
\label{apiRest:servicio-de-actualizacion-de-ubicacion}
\begin{Verbatim}[commandchars=\\\{\}]
URL
        URL: http://\PYGZlt{}appserver\PYGZgt{}/api/location
        HTTP VERB
        POST
        PARAMETROS
        \PYGZob{}
              \PYGZdq{}token\PYGZdq{}: “hash”,
              \PYGZdq{}latitude\PYGZdq{}: “\PYGZhy{}34.595241”,
              \PYGZdq{}longitude\PYGZdq{}: “\PYGZhy{}58.402460”
\PYGZcb{}
        RESPUESTA
        HTTP CODE 200
\PYGZob{}
         \PYGZdq{}data\PYGZdq{}:
            \PYGZob{}
               \PYGZdq{}message\PYGZdq{}:\PYGZdq{}Ubicacion de usuario actualizada\PYGZdq{},
               \PYGZdq{}status\PYGZdq{}:\PYGZdq{}OK\PYGZdq{}
            \PYGZcb{}
         ,
         \PYGZdq{}errors\PYGZdq{}:[
         ]
\PYGZcb{}


HTTP CODE 400
\PYGZob{}
         \PYGZdq{}data\PYGZdq{}:[
         ],
         \PYGZdq{}errors\PYGZdq{}:[
\PYGZob{}
               \PYGZdq{}message\PYGZdq{}:\PYGZdq{}Latitud o longitud vacios\PYGZdq{},
               \PYGZdq{}status\PYGZdq{}:\PYGZdq{}ERROR\PYGZdq{}
            \PYGZcb{}


         ]
\PYGZcb{}
\end{Verbatim}


\subsection{Servicio de vista de mis contactos}
\label{apiRest:servicio-de-vista-de-mis-contactos}
\begin{Verbatim}[commandchars=\\\{\}]
URL
        URL: http://\PYGZlt{}appserver\PYGZgt{}/api/get\PYGZus{}contacts
        HTTP VERB
        GET
        PARAMETROS
        \PYGZob{}
              \PYGZdq{}token\PYGZdq{}: “hash”
\PYGZcb{}
        RESPUESTA
        \PYGZob{}
         \PYGZdq{}data\PYGZdq{}:[
               \PYGZob{}
                  \PYGZdq{}distance”:”3.978116”,
                  \PYGZdq{}fb\PYGZus{}id\PYGZdq{}:\PYGZdq{}tomas\PYGZdq{},
                  \PYGZdq{}is\PYGZus{}contact\PYGZdq{}:\PYGZdq{}true\PYGZdq{},
                  \PYGZdq{}name\PYGZdq{}:\PYGZdq{}Tomas mussi\PYGZdq{},
                  \PYGZdq{}photo\PYGZdq{}:\PYGZdq{}bas64encode..==\PYGZdq{}
               \PYGZcb{}
         ],
         \PYGZdq{}errors\PYGZdq{}:[


         ]
\PYGZcb{}
\end{Verbatim}


\subsection{Servicio de agregar contacto}
\label{apiRest:servicio-de-agregar-contacto}
\begin{Verbatim}[commandchars=\\\{\}]
URL
        URL: http://\PYGZlt{}appserver\PYGZgt{}/api/contact/
        HTTP VERB
        POST
        PARAMETROS
        \PYGZob{}
              \PYGZdq{}token\PYGZdq{}: “hash”,
              “contact\PYGZus{}fb\PYGZus{}id” : “a\PYGZus{}user\PYGZus{}fb\PYGZus{}id”
\PYGZcb{}
        RESPUESTA
        \PYGZob{}
         \PYGZdq{}data\PYGZdq{}:[
            \PYGZob{}
               \PYGZdq{}message\PYGZdq{}:\PYGZdq{}Enviada solicitud a contacto\PYGZdq{},
               \PYGZdq{}status\PYGZdq{}:\PYGZdq{}OK\PYGZdq{}
            \PYGZcb{}
         ],
         \PYGZdq{}errors\PYGZdq{}:[
         ]
\PYGZcb{}


        RESPUESTA ERROR
        \PYGZob{}
         “data”:[],
         “errors”: [
              “status” : “ERROR”,
              “message” : “El usuario no existe”
         ]
\PYGZcb{}
\end{Verbatim}


\subsection{Servicio de vista respuesta de solicitudes de contactos}
\label{apiRest:servicio-de-vista-respuesta-de-solicitudes-de-contactos}
\begin{Verbatim}[commandchars=\\\{\}]
URL
        URL: http://\PYGZlt{}appserver\PYGZgt{}/api/contact/response
        HTTP VERB
        POST
        PARAMETROS
        \PYGZob{}
              \PYGZdq{}token\PYGZdq{}: “hash”,
              “contact\PYGZus{}fb\PYGZus{}id” : “a\PYGZus{}user\PYGZus{}fb\PYGZus{}id”,
              \PYGZdq{}accept\PYGZdq{}: “true” / “false”
\PYGZcb{}
        RESPUESTA
        \PYGZob{}
         \PYGZdq{}data\PYGZdq{}:[
            \PYGZob{}
               \PYGZdq{}message\PYGZdq{}:\PYGZdq{}Respondida solicitud a contacto\PYGZdq{},
               \PYGZdq{}status\PYGZdq{}:\PYGZdq{}OK\PYGZdq{}
            \PYGZcb{}
         ],
         \PYGZdq{}errors\PYGZdq{}:[
         ]
\PYGZcb{}


        RESPUESTA ERROR
        \PYGZob{}
         “data”:[],
         “errors”: [
              “status” : “ERROR”,
              “message” : “El usuario no existe”
         ]
\PYGZcb{}
\end{Verbatim}


\subsection{Servicio de vista de solicitudes pendientes de contactos}
\label{apiRest:servicio-de-vista-de-solicitudes-pendientes-de-contactos}
\begin{Verbatim}[commandchars=\\\{\}]
URL
        URL: http://\PYGZlt{}appserver\PYGZgt{}/api/contact/
        HTTP VERB
        GET
        PARAMETROS
        \PYGZob{}
              \PYGZdq{}token\PYGZdq{}: “hash”
\PYGZcb{}
        RESPUESTA
        \PYGZob{}
         \PYGZdq{}data\PYGZdq{}:[
            [
               \PYGZob{}
                  \PYGZdq{}fb\PYGZus{}id\PYGZdq{}:\PYGZdq{}tomas\PYGZdq{},
                  \PYGZdq{}is\PYGZus{}contact\PYGZdq{}:\PYGZdq{}false\PYGZdq{},
                  \PYGZdq{}name\PYGZdq{}:\PYGZdq{}Tomas mussi\PYGZdq{},
                  \PYGZdq{}photo\PYGZdq{}:\PYGZdq{}bas64encode..==\PYGZdq{}
               \PYGZcb{},
               \PYGZob{}
                  \PYGZdq{}fb\PYGZus{}id\PYGZdq{}:\PYGZdq{}luis\PYGZdq{},
                  \PYGZdq{}is\PYGZus{}contact\PYGZdq{}:\PYGZdq{}false\PYGZdq{},
                  \PYGZdq{}name\PYGZdq{}:\PYGZdq{}Luis Arancibia\PYGZdq{},
                  \PYGZdq{}photo\PYGZdq{}:\PYGZdq{}bas64encode..==\PYGZdq{}
               \PYGZcb{}
            ]
         ],
         \PYGZdq{}errors\PYGZdq{}:[


         ]
\PYGZcb{}


        RESPUESTA ERROR
        \PYGZob{}
         “data”:[],
         “errors”: [
              “status” : “ERROR”,
              “message” : “El usuario no existe”
         ]
\PYGZcb{}
\end{Verbatim}


\subsection{Servicio de recomendación}
\label{apiRest:servicio-de-recomendacion}
\begin{Verbatim}[commandchars=\\\{\}]
URL
        URL: http://\PYGZlt{}appserver\PYGZgt{}/api/vote
        HTTP VERB
        POST
        PARAMETROS
        \PYGZob{}
              \PYGZdq{}token\PYGZdq{}: “hash”,
              “contact\PYGZus{}fb\PYGZus{}id” : “a\PYGZus{}user\PYGZus{}fb\PYGZus{}id”
\PYGZcb{}
        RESPUESTA
        \PYGZob{}
         \PYGZdq{}data\PYGZdq{}:[
            \PYGZob{}
               \PYGZdq{}message\PYGZdq{}:\PYGZdq{}Enviada votación a contacto\PYGZdq{},
               \PYGZdq{}status\PYGZdq{}:\PYGZdq{}OK\PYGZdq{}
            \PYGZcb{}
         ],
         \PYGZdq{}errors\PYGZdq{}:[


         ]
\PYGZcb{}


        RESPUESTA ERROR
        \PYGZob{}
         \PYGZdq{}data\PYGZdq{}:[
         ],
         \PYGZdq{}errors\PYGZdq{}:[
               \PYGZdq{}message\PYGZdq{}:\PYGZdq{}Usuario [a\PYGZus{}user\PYGZus{}fb\PYGZus{}id] no existe o no es valido\PYGZdq{},
               \PYGZdq{}status\PYGZdq{}:\PYGZdq{}ERROR\PYGZdq{}
         ]
\PYGZcb{}
\end{Verbatim}


\subsection{Servicio de consulta de usuarios más populares}
\label{apiRest:servicio-de-consulta-de-usuarios-mas-populares}
\begin{Verbatim}[commandchars=\\\{\}]
URL
        URL: http://\PYGZlt{}appserver\PYGZgt{}/api/vote/popular
        HTTP VERB
        GET
        PARAMETROS
        \PYGZob{}
              \PYGZdq{}token\PYGZdq{}: “hash”,
\PYGZcb{}
        RESPUESTA
        \PYGZob{}
         \PYGZdq{}data\PYGZdq{}:[
            [
               \PYGZob{}
                  \PYGZdq{}fb\PYGZus{}id\PYGZdq{}:\PYGZdq{}tomas\PYGZdq{},
                  “is\PYGZus{}contact” : “true”,
                  “name” : “Tomas Mussi”,
                  “photo” : “dsjagkjasg…==”,
                  \PYGZdq{}votes\PYGZdq{}:1
               \PYGZcb{},
               \PYGZob{}
                  \PYGZdq{}fb\PYGZus{}id\PYGZdq{}:\PYGZdq{}aran.com.ar@gmail.com\PYGZdq{},
                  “is\PYGZus{}contact” : “false”,
                  “name” : “Luis Arancibia”,
                  “photo” : “dsjagkjasg…==”,
                  \PYGZdq{}votes\PYGZdq{}:0
               \PYGZcb{},
               \PYGZob{}
                  \PYGZdq{}fb\PYGZus{}id\PYGZdq{}:\PYGZdq{}a\PYGZhy{}fb\PYGZhy{}user\PYGZhy{}id\PYGZdq{},
                  “is\PYGZus{}contact” : “false”,
                  “name” : “User Name”,
                  “photo” : “dsjagkjasg…==”,
                  \PYGZdq{}votes\PYGZdq{}:0
               \PYGZcb{}
            ]
         ],
         \PYGZdq{}errors\PYGZdq{}:[
         ]
\PYGZcb{}
\end{Verbatim}


\subsection{Servicio de conversaciones}
\label{apiRest:servicio-de-conversaciones}
\begin{Verbatim}[commandchars=\\\{\}]
URL
        URL: http://\PYGZlt{}appserver\PYGZgt{}/api/message/
        HTTP VERB
        GET
        PARAMETROS
        \PYGZob{}
              \PYGZdq{}token\PYGZdq{}: “hash”,
              “contact\PYGZus{}fb\PYGZus{}id”: “tomas”
\PYGZcb{}
        RESPUESTA
        \PYGZob{}
         \PYGZdq{}data\PYGZdq{}:
            [
                 \PYGZob{}     \PYGZdq{}message\PYGZdq{}:\PYGZdq{}como va?\PYGZdq{},
                     \PYGZdq{}receiver\PYGZus{}id\PYGZdq{}:\PYGZdq{}tomas\PYGZdq{},
                     \PYGZdq{}sender\PYGZus{}id\PYGZdq{}:\PYGZdq{}aran.com.ar@gmail.com\PYGZdq{},
                     \PYGZdq{}timestamp\PYGZdq{}:\PYGZdq{}2016\PYGZhy{}11\PYGZhy{}05 18:48:03\PYGZdq{}
                  \PYGZcb{},
                  \PYGZob{}
                     \PYGZdq{}message\PYGZdq{}:\PYGZdq{}hola\PYGZdq{},
                     \PYGZdq{}receiver\PYGZus{}id\PYGZdq{}:\PYGZdq{}aran.com.ar@gmail.com\PYGZdq{},
                     \PYGZdq{}sender\PYGZus{}id\PYGZdq{}:\PYGZdq{}tomas\PYGZdq{},
                     \PYGZdq{}timestamp\PYGZdq{}:\PYGZdq{}2016\PYGZhy{}11\PYGZhy{}05 18:44:22\PYGZdq{}
                  \PYGZcb{}
             ]
         ,
         \PYGZdq{}errors\PYGZdq{}:[


         ]
\PYGZcb{}
\end{Verbatim}


\subsection{Servicio de envio de mensajes}
\label{apiRest:servicio-de-envio-de-mensajes}
\begin{Verbatim}[commandchars=\\\{\}]
URL
        URL: http://\PYGZlt{}appserver\PYGZgt{}/api/message
        HTTP VERB
        POST
        PARAMETROS
        \PYGZob{}
              \PYGZdq{}token\PYGZdq{}: “hash”,
              “contact\PYGZus{}fb\PYGZus{}id” : “tomas”,
              “message” : “Hola, como te va?”
\PYGZcb{}
        RESPUESTA
        \PYGZob{}
         \PYGZdq{}data\PYGZdq{}:
            \PYGZob{}
               \PYGZdq{}message\PYGZdq{}:\PYGZdq{}Mensaje enviado exitosamente\PYGZdq{},
               \PYGZdq{}status\PYGZdq{}:\PYGZdq{}OK\PYGZdq{}
            \PYGZcb{}
        ,
         \PYGZdq{}errors\PYGZdq{}:\PYGZob{}\PYGZcb{}
\PYGZcb{}
\end{Verbatim}


\subsection{Servicio de consulta de perfil de otros usuarios}
\label{apiRest:servicio-de-consulta-de-perfil-de-otros-usuarios}
\begin{Verbatim}[commandchars=\\\{\}]
URL
        URL: http://\PYGZlt{}appserver\PYGZgt{}/api/profile/others
        HTTP VERB
        GET
        PARAMETROS
        \PYGZob{}
              \PYGZdq{}token\PYGZdq{}: “hash”,
              \PYGZdq{}contact\PYGZus{}fb\PYGZus{}id\PYGZdq{}: “other\PYGZus{}fb\PYGZus{}id”
\PYGZcb{}
        RESPUESTA
        \PYGZob{}
         \PYGZdq{}data\PYGZdq{}:[
            \PYGZob{}
               \PYGZdq{}user\PYGZdq{}:\PYGZob{}
                  \PYGZdq{}city\PYGZdq{}:\PYGZdq{}\PYGZdq{},
                  \PYGZdq{}contacts\PYGZdq{}:1,
                  \PYGZdq{}dob\PYGZdq{}:\PYGZdq{}11/07/1991\PYGZdq{},
                  \PYGZdq{}email\PYGZdq{}:\PYGZdq{}\PYGZdq{},
                  \PYGZdq{}job\PYGZus{}positions\PYGZdq{}:[


                  ],
                  \PYGZdq{}name\PYGZdq{}:\PYGZdq{}Tomas Mussi\PYGZdq{},
                  \PYGZdq{}profile\PYGZus{}photo\PYGZdq{}:\PYGZdq{}\PYGZdq{},
                  \PYGZdq{}requests\PYGZdq{}:[


                  ],
                  \PYGZdq{}skills\PYGZdq{}:[


                  ],
                  \PYGZdq{}summary\PYGZdq{}:\PYGZdq{}A summary\PYGZdq{}
               \PYGZcb{}
            \PYGZcb{}[a]
         ],
         \PYGZdq{}errors\PYGZdq{}:[
         ]
\PYGZcb{}
\end{Verbatim}


\subsection{Servicio de consulta de perfil}
\label{apiRest:servicio-de-consulta-de-perfil}
\begin{Verbatim}[commandchars=\\\{\}]
URL
        URL: http://\PYGZlt{}appserver\PYGZgt{}/api/profile
        HTTP VERB
        GET
        PARAMETROS
        \PYGZob{}
              \PYGZdq{}token\PYGZdq{}: “hash”
\PYGZcb{}
        RESPUESTA
        \PYGZob{}
         \PYGZdq{}data\PYGZdq{}[
            \PYGZob{}
               \PYGZdq{}user\PYGZdq{}:\PYGZob{}
                  \PYGZdq{}city\PYGZdq{} : \PYGZdq{}Ciudad de Buenos Aires\PYGZdq{},
                  \PYGZdq{}contacts\PYGZdq{}:1,
                  \PYGZdq{}dob\PYGZdq{} : \PYGZdq{}11/07/1991\PYGZdq{},
                  \PYGZdq{}email\PYGZdq{}:\PYGZdq{}tomasmussi@gmail.com\PYGZdq{},
                  \PYGZdq{}job\PYGZus{}positions\PYGZdq{}:[


                  ],
                  \PYGZdq{}name\PYGZdq{}:\PYGZdq{}\PYGZdq{},
                  \PYGZdq{}profile\PYGZus{}photo\PYGZdq{} : \PYGZdq{}QURQIEdtYkgK...mdHVuZw==\PYGZdq{},
                  \PYGZdq{}requests\PYGZdq{}:[
                  ],
                  \PYGZdq{}skills\PYGZdq{}:[
                     \PYGZdq{}c\PYGZdq{},
                     \PYGZdq{}c++\PYGZdq{}
                  ],
                 \PYGZdq{}summary\PYGZdq{} : \PYGZdq{}Estudiante de ingenieria informatica de la UBA.\PYGZdq{}
               \PYGZcb{}
            \PYGZcb{}
         ],[b]
         \PYGZdq{}errors\PYGZdq{}:[
         ]
\PYGZcb{}
\end{Verbatim}


\subsection{Servicio de administración de perfil de usuario}
\label{apiRest:servicio-de-administracion-de-perfil-de-usuario}
\begin{Verbatim}[commandchars=\\\{\}]
URL
        URL: http://\PYGZlt{}appserver\PYGZgt{}/api/profile/
        HTTP VERB
        POST
        PARAMETROS
        \PYGZob{}
              \PYGZdq{}token\PYGZdq{}: “hash”,
              \PYGZdq{}name\PYGZdq{}: \PYGZdq{}Tomas Mussi Reyero”,
              “summary” : “Estudiante de ingenieria informatica de la UBA.”,
              “profile\PYGZus{}photo” : “QURQIEdtYkgK...mdHVuZw==”,
\PYGZcb{}
        RESPUESTA
        \PYGZob{}
        \PYGZdq{}data\PYGZdq{}: [
          \PYGZob{}
            \PYGZdq{}message\PYGZdq{}: \PYGZdq{}Usuario modificado exitosamente\PYGZdq{},
            \PYGZdq{}status\PYGZdq{}: \PYGZdq{}OK\PYGZdq{}
          \PYGZcb{}
        ],
        \PYGZdq{}errors\PYGZdq{}: []
\PYGZcb{}
\end{Verbatim}


\subsection{Servicio de administración de skills de usuario}
\label{apiRest:servicio-de-administracion-de-skills-de-usuario}
\begin{Verbatim}[commandchars=\\\{\}]
URL
        URL: http://\PYGZlt{}appserver\PYGZgt{}/api/skills
        HTTP VERB
        GET
        PARAMETROS


        RESPUESTA
        \PYGZob{}
              \PYGZdq{}skills\PYGZdq{}: [\PYGZob{}
                      \PYGZdq{}name\PYGZdq{}: \PYGZdq{}c\PYGZdq{},
                      \PYGZdq{}description\PYGZdq{}: \PYGZdq{}c programming language\PYGZdq{},
                      \PYGZdq{}category\PYGZdq{}: \PYGZdq{}software\PYGZdq{}
              \PYGZcb{}, \PYGZob{}
                      \PYGZdq{}name\PYGZdq{}: \PYGZdq{}PMI\PYGZdq{},
                      \PYGZdq{}description\PYGZdq{}: \PYGZdq{}Project Management Institute\PYGZdq{},
                      \PYGZdq{}category\PYGZdq{}: \PYGZdq{}management\PYGZdq{}
              \PYGZcb{}],
              \PYGZdq{}metadata\PYGZdq{}: \PYGZob{}
                      \PYGZdq{}version\PYGZdq{}: \PYGZdq{}0.1\PYGZdq{},
                      \PYGZdq{}count\PYGZdq{}: 2
              \PYGZcb{}
\PYGZcb{}
\end{Verbatim}


\subsection{Servicio de administración de agregar un skill al usuario}
\label{apiRest:servicio-de-administracion-de-agregar-un-skill-al-usuario}
\begin{Verbatim}[commandchars=\\\{\}]
URL
        URL: http://\PYGZlt{}appserver\PYGZgt{}/api/skills
        HTTP VERB
        POST
        PARAMETROS
        \PYGZob{}
              \PYGZdq{}token\PYGZdq{}: “hash”,
              \PYGZdq{}skill\PYGZdq{}: “c”
\PYGZcb{}
        RESPUESTA
        \PYGZob{}
        \PYGZdq{}data\PYGZdq{}: [
          \PYGZob{}
            \PYGZdq{}message\PYGZdq{}: \PYGZdq{}Skill agregado\PYGZdq{},
            \PYGZdq{}status\PYGZdq{}: \PYGZdq{}OK\PYGZdq{}
          \PYGZcb{}
        ],
        \PYGZdq{}errors\PYGZdq{}: []
\PYGZcb{}
\end{Verbatim}


\subsection{Servicio de administración de eliminación de un skill de usuario}
\label{apiRest:servicio-de-administracion-de-eliminacion-de-un-skill-de-usuario}
\begin{Verbatim}[commandchars=\\\{\}]
URL
        URL: http://\PYGZlt{}appserver\PYGZgt{}/api/skills
        HTTP VERB
        DELETE
        PARAMETROS
        \PYGZob{}
              \PYGZdq{}token\PYGZdq{}: “hash”,
              \PYGZdq{}skill\PYGZdq{}: “c”
\PYGZcb{}
        RESPUESTA
        \PYGZob{}
        \PYGZdq{}data\PYGZdq{}: [
          \PYGZob{}
            \PYGZdq{}message\PYGZdq{}: \PYGZdq{}Skill eliminado\PYGZdq{},
            \PYGZdq{}status\PYGZdq{}: \PYGZdq{}OK\PYGZdq{}
          \PYGZcb{}
        ],
        \PYGZdq{}errors\PYGZdq{}: []
\PYGZcb{}
\end{Verbatim}


\subsection{Servicio de administración de obtención de un skill en particular}
\label{apiRest:servicio-de-administracion-de-obtencion-de-un-skill-en-particular}
\begin{Verbatim}[commandchars=\\\{\}]
URL
        URL: http://\PYGZlt{}appserver\PYGZgt{}/api/skill
        HTTP VERB
        GET
        PARAMETROS
        \PYGZob{}
              \PYGZdq{}token\PYGZdq{}: “hash”,
              \PYGZdq{}name\PYGZdq{}: “c”,
\PYGZcb{}
        RESPUESTA
        \PYGZob{}
        \PYGZdq{}data\PYGZdq{}: [
 \PYGZob{}
              \PYGZdq{}name\PYGZdq{}: \PYGZdq{}c\PYGZdq{},
              \PYGZdq{}description\PYGZdq{}: \PYGZdq{}c programming language\PYGZdq{},
              \PYGZdq{}category\PYGZdq{}: \PYGZdq{}software\PYGZdq{}
\PYGZcb{}
        ],
        \PYGZdq{}errors\PYGZdq{}: []
\PYGZcb{}
\end{Verbatim}


\subsection{Servicio de administración de job positions de usuario}
\label{apiRest:servicio-de-administracion-de-job-positions-de-usuario}
\begin{Verbatim}[commandchars=\\\{\}]
URL
        URL: http://\PYGZlt{}appserver\PYGZgt{}/api/job\PYGZus{}positions
        HTTP VERB
        GET
        PARAMETROS


        RESPUESTA
        \PYGZob{}
              \PYGZdq{}job\PYGZus{}positions\PYGZdq{}: [\PYGZob{}
                      \PYGZdq{}name\PYGZdq{}: \PYGZdq{}developer\PYGZdq{},
                      \PYGZdq{}description\PYGZdq{}: \PYGZdq{} a software developer\PYGZdq{}
                      \PYGZdq{}category\PYGZdq{}: \PYGZdq{}software\PYGZdq{}
              \PYGZcb{}, \PYGZob{}


                      \PYGZdq{}name\PYGZdq{}: \PYGZdq{}project manager\PYGZdq{},
                      \PYGZdq{}description\PYGZdq{}: \PYGZdq{} a project manager\PYGZdq{}
                      \PYGZdq{}category\PYGZdq{}: \PYGZdq{}management\PYGZdq{}
              \PYGZcb{}, \PYGZob{}
                      \PYGZdq{}name\PYGZdq{}: \PYGZdq{}dj\PYGZdq{},
                      \PYGZdq{}category\PYGZdq{}: \PYGZdq{}music\PYGZdq{}
              \PYGZcb{}],
              \PYGZdq{}metadata\PYGZdq{}: \PYGZob{}
                      \PYGZdq{}version\PYGZdq{}: \PYGZdq{}0.1\PYGZdq{},
                      \PYGZdq{}count\PYGZdq{}: 3
              \PYGZcb{}
\PYGZcb{}
\end{Verbatim}


\subsection{Servicio de administración de agregar un job positions al usuario}
\label{apiRest:servicio-de-administracion-de-agregar-un-job-positions-al-usuario}
\begin{Verbatim}[commandchars=\\\{\}]
URL
        URL: http://\PYGZlt{}appserver\PYGZgt{}/api/job\PYGZus{}positions
        HTTP VERB
        POST
        PARAMETROS
        \PYGZob{}
              \PYGZdq{}token\PYGZdq{}: “hash”,
              \PYGZdq{}job\PYGZdq{}: “developer”
\PYGZcb{}
        RESPUESTA
        \PYGZob{}
        \PYGZdq{}data\PYGZdq{}: [
          \PYGZob{}
            \PYGZdq{}message\PYGZdq{}: \PYGZdq{}Job position agregado\PYGZdq{},
            \PYGZdq{}status\PYGZdq{}: \PYGZdq{}OK\PYGZdq{}
          \PYGZcb{}
        ],
        \PYGZdq{}errors\PYGZdq{}: []
\PYGZcb{}
\end{Verbatim}


\subsection{Servicio de administración de eliminación de un job position de usuario}
\label{apiRest:servicio-de-administracion-de-eliminacion-de-un-job-position-de-usuario}
\begin{Verbatim}[commandchars=\\\{\}]
URL
        URL: http://\PYGZlt{}appserver\PYGZgt{}/api/job\PYGZus{}positions
        HTTP VERB
        DELETE
        PARAMETROS
        \PYGZob{}
              \PYGZdq{}token\PYGZdq{}: “hash”,
              \PYGZdq{}job\PYGZdq{}: “developer”
\PYGZcb{}
        RESPUESTA
        \PYGZob{}
        \PYGZdq{}data\PYGZdq{}: [
          \PYGZob{}
            \PYGZdq{}message\PYGZdq{}: \PYGZdq{}Job position eliminado\PYGZdq{},
            \PYGZdq{}status\PYGZdq{}: \PYGZdq{}OK\PYGZdq{}
          \PYGZcb{}
        ],
        \PYGZdq{}errors\PYGZdq{}: []
\PYGZcb{}
\end{Verbatim}


\subsection{Servicio de administración de obtención de un jobPosition en particular}
\label{apiRest:servicio-de-administracion-de-obtencion-de-un-jobposition-en-particular}
\begin{Verbatim}[commandchars=\\\{\}]
URL
        URL: http://\PYGZlt{}appserver\PYGZgt{}/api/job\PYGZus{}position
        HTTP VERB
        GET
        PARAMETROS
        \PYGZob{}
              \PYGZdq{}token\PYGZdq{}: “hash”,
              \PYGZdq{}name\PYGZdq{}: “developer”,
\PYGZcb{}
        RESPUESTA
        \PYGZob{}
        \PYGZdq{}data\PYGZdq{}: [
 \PYGZob{}
              \PYGZdq{}name\PYGZdq{}: \PYGZdq{}developer\PYGZdq{},
              \PYGZdq{}description\PYGZdq{}: \PYGZdq{} a software developer\PYGZdq{}
              \PYGZdq{}category\PYGZdq{}: \PYGZdq{}software\PYGZdq{}
\PYGZcb{}
        ],
        \PYGZdq{}errors\PYGZdq{}: []
\PYGZcb{}
\end{Verbatim}


\subsection{Servicio de registración de token Firebase Cloud Messages}
\label{apiRest:servicio-de-registracion-de-token-firebase-cloud-messages}
\begin{Verbatim}[commandchars=\\\{\}]
URL
        URL: http://\PYGZlt{}appserver\PYGZgt{}/api/token\PYGZus{}FCM
        HTTP VERB
        POST
        PARAMETROS
        \PYGZob{}
              \PYGZdq{}token\PYGZdq{}: “hash”,
              \PYGZdq{}token\PYGZus{}FCM\PYGZdq{}: “hash”,
\PYGZcb{}
        RESPUESTA
        \PYGZob{}
        \PYGZdq{}data\PYGZdq{}:
 \PYGZob{}
              \PYGZdq{}message\PYGZdq{}: \PYGZdq{}El token ha sido registrado satisfactoriamente\PYGZdq{},
              \PYGZdq{}status\PYGZdq{}: \PYGZdq{}OK\PYGZdq{}
\PYGZcb{},
        \PYGZdq{}errors\PYGZdq{}: []
\PYGZcb{}
\end{Verbatim}


\section{Pruebas Unitarias y funcionales}
\label{appServer:pruebas-unitarias-y-funcionales}

\subsection{Objetivo}
\label{testing:objetivo}\label{testing::doc}
Exponer al programador cómo se realizaron tests sobre la aplicación, describir paso a paso la forma de ejecutar los tests y explicar cómo mantener y mejorar los mismos.


\subsection{Realización}
\label{testing:realizacion}
Los tests están realizados para verificar la funcionalidad principalmente del app server, desarrollado en C++. Dichos tests se realizaron con el framework de testing de Google {[}1{]}.
Los mismos verifican el comportamiento unitario de las clases del servidor.
Los tests de funcionalidad de la ApiREST se desarrollaron en python.


\subsection{Ejecución}
\label{testing:ejecucion}
Para ejecutar las pruebas, deben seguirse los pasos de compilación del servidor especificados en la documentación técnica del app server {[}2{]}, pero en vez de ejecutar el binario del servidor src/server, se debe ejecutar el binario generado testing que se encuentra dentro de la carpeta test.


\subsection{Coverage}
\label{testing:coverage}
Para medir el coverage se debe ejecutar el script ./coverage.sh. Si se tiene instalado el firefox se abre automaticamente, sino se debe abrir la web generada en el archivo index en la carpeta coverage dentro de este mismo directorio.
Este test queda a la espera de que uno cierre el test dado que se deben ejecutar las pruebas de la ApiRest y las mismas necesitan que el AppServer esté corriendo.
Se debe ir al directorio /server/test/ y ejectutar bash ./python/test.sh cuando finalicen los test se puede finalizar el Appserver.


\subsection{Referencias}
\label{testing:referencias}\begin{itemize}
\item {} 
{[}1{]}: Google test framework para c/c++: \url{https://github.com/google/googletest}

\item {} 
{[}2{]}: Documentación técnica app server: \url{https://docs.google.com/document/d/1Db988RfcEOvXmyQ6PIe9AO\_NscSqZUgfGTnTWRT6-60/edit\#heading=h.abohzj50eg4c}

\end{itemize}


\chapter{Aplicación Android}
\label{index:aplicacion-android}

\section{Objetivo}
\label{appAndroid:objetivo}\label{appAndroid::doc}
Documentar información técnica de la aplicación android para poder compilar, ejecutar, debuggear, extender y mejorar la misma


\section{Requisitos}
\label{appAndroid:requisitos}
La aplicación fue desarrollada en Android studio y su SDK {[}1{]}

Otros componentes adicionales que se utilizaron fueron
\begin{itemize}
\item {} 
\textbf{Firebase} {[}2{]} como una plataforma movil que se encarga de la autentificación de los usuarios mediante cuentas de facebook, gmail.

\item {} 
\textbf{Firebase Cloud Messages} {[}3{]} servicio provisto por Firebase para proveer a nuestra app de notificaciones

\end{itemize}


\section{Distribución}
\label{appAndroid:distribucion}
El código puede ser descargado de la página oficial de github {[}4{]}.


\section{Sistema operativo}
\label{appAndroid:sistema-operativo}
El desarrollo de la aplicación se ha realizado en Android Studio siendo ejecutado el mismo bajo el sistema operativo Linux. De querer desarrollar en Windows o Mac, no se garantiza el correcto funcionamiento de la misma

\emph{La aplicación funciona una vez compilada solamente en sistemas operativos Android}


\section{Compilación}
\label{appAndroid:compilacion}
La aplicación ha sido compilada e instalada en celulares y emuladores provisto por android studio con una API mayor a 21


\section{Descripción del proyecto}
\label{appAndroid:descripcion-del-proyecto}
Está formado por los siguientes paquetes.
\begin{itemize}
\item {} 
\textbf{activities}: Contiene el comportamiento de cada una de las pantallas.

\item {} 
\textbf{adapters}: Utilizado para definir los adapters que contiene la aplicación.

\item {} 
\textbf{domain}: Contiene las clases del dominio del problema

\item {} 
\textbf{rest\_dto}: Contiene las clases que modelan la serialización e hidratación de los datos desde un json a clases de dominio.

\item {} 
\textbf{services}: Contiene cada uno de los servicios que interactuan con el appServer.

\end{itemize}


\section{Explicación de como agregar un servicio}
\label{appAndroid:explicacion-de-como-agregar-un-servicio}

\subsection{Agregar una clase de servicio nueva}
\label{appAndroid:agregar-una-clase-de-servicio-nueva}
En este caso agregamos de ejemplo un servicio existente en la aplicación.
\begin{itemize}
\item {} 
Se debe crear un servicio que herede de AbstractServices.

\end{itemize}

\begin{Verbatim}[commandchars=\\\{\}]
\PYG{k+kd}{public} \PYG{k+kd}{class} \PYG{n+nc}{GetSkillServices} \PYG{k+kd}{extends} \PYG{n}{AbstractServices} \PYG{o}{\PYGZob{}}
          \PYG{k+kd}{private} \PYG{k+kd}{static} \PYG{k+kd}{final} \PYG{n}{String} \PYG{n}{service\PYGZus{}name} \PYG{o}{=} \PYG{l+s}{\PYGZdq{}api/skill\PYGZdq{}}\PYG{o}{;}
\PYG{o}{.}\PYG{o}{.}\PYG{o}{.}\PYG{o}{\PYGZcb{}}
\end{Verbatim}

donde \emph{service\_name} indica el nombre del servicio a consultar

\begin{Verbatim}[commandchars=\\\{\}]
\PYG{k+kd}{public} \PYG{n}{Skill} \PYG{n+nf}{get}\PYG{o}{(}\PYG{n}{String} \PYG{n}{nameSkill}\PYG{o}{)} \PYG{o}{\PYGZob{}}
    \PYG{n}{String} \PYG{n}{query} \PYG{o}{=} \PYG{k}{this}\PYG{o}{.}\PYG{n+na}{getQueryBy}\PYG{o}{(}\PYG{n}{nameSkill}\PYG{o}{)}\PYG{o}{;}
\end{Verbatim}

hay que definir el metodo get() que es el que nos va a permitir obtener el objeto hidratado ya con el estado valido segun la consulta realizada

\begin{Verbatim}[commandchars=\\\{\}]
\PYG{k+kd}{public} \PYG{n}{Skill} \PYG{n+nf}{get}\PYG{o}{(}\PYG{n}{String} \PYG{n}{nameSkill}\PYG{o}{)} \PYG{o}{\PYGZob{}}
    \PYG{o}{.}\PYG{o}{.}\PYG{o}{.}
    \PYG{n}{SkillDTO} \PYG{n}{skillDTO} \PYG{o}{=} \PYG{o}{(}\PYG{n}{SkillDTO}\PYG{o}{)} \PYG{n}{getDataOftDTO}\PYG{o}{(}\PYG{n}{query}\PYG{o}{,} \PYG{n}{SkillDTO}\PYG{o}{.}\PYG{n+na}{class}\PYG{o}{)}\PYG{o}{;}
    \PYG{k}{return} \PYG{n}{skillDTO}\PYG{o}{.}\PYG{n+na}{getData}\PYG{o}{(}\PYG{o}{)}\PYG{o}{;}
\PYG{o}{\PYGZcb{}}
\end{Verbatim}

Vemos como hay que instanciar un dto, el mismo nos va a permitir según la query que obtuvimos generar el objeto correspondiente.

\begin{Verbatim}[commandchars=\\\{\}]
\PYG{n+nd}{@Override}
          \PYG{k+kd}{protected} \PYG{n}{String} \PYG{n+nf}{getQueryBy}\PYG{o}{(}\PYG{n}{String}\PYG{o}{.}\PYG{o}{.}\PYG{o}{.} \PYG{n}{params}\PYG{o}{)} \PYG{o}{\PYGZob{}}
              \PYG{n}{String} \PYG{n}{nameSkill} \PYG{o}{=} \PYG{n}{params} \PYG{o}{[}\PYG{l+m+mi}{0}\PYG{o}{]}\PYG{o}{;}

              \PYG{n}{String} \PYG{n}{url} \PYG{o}{=} \PYG{n}{urlBase}\PYG{o}{;}
              \PYG{n}{StringBuffer} \PYG{n}{urlStringBuffer} \PYG{o}{=} \PYG{k}{new} \PYG{n}{StringBuffer}\PYG{o}{(}\PYG{n}{url}\PYG{o}{)}\PYG{o}{;}
              \PYG{n}{urlStringBuffer}\PYG{o}{.}\PYG{n+na}{append}\PYG{o}{(}\PYG{n}{service\PYGZus{}name}\PYG{o}{)}\PYG{o}{;}
              \PYG{n}{urlStringBuffer}\PYG{o}{.}\PYG{n+na}{append}\PYG{o}{(}\PYG{l+s}{\PYGZdq{}?\PYGZdq{}}\PYG{o}{)}\PYG{o}{;}
              \PYG{n}{urlStringBuffer}\PYG{o}{.}\PYG{n+na}{append}\PYG{o}{(}\PYG{l+s}{\PYGZdq{}token=\PYGZdq{}}\PYG{o}{)}\PYG{o}{;}
              \PYG{n}{urlStringBuffer}\PYG{o}{.}\PYG{n+na}{append}\PYG{o}{(}\PYG{n}{api\PYGZus{}security}\PYG{o}{)}\PYG{o}{;}
              \PYG{n}{urlStringBuffer}\PYG{o}{.}\PYG{n+na}{append}\PYG{o}{(}\PYG{l+s}{\PYGZdq{}\PYGZam{}name=\PYGZdq{}}\PYG{o}{)}\PYG{o}{;}
              \PYG{n}{urlStringBuffer}\PYG{o}{.}\PYG{n+na}{append}\PYG{o}{(}\PYG{n}{nameSkill}\PYG{o}{)}\PYG{o}{;}
              \PYG{n}{System}\PYG{o}{.}\PYG{n+na}{out}\PYG{o}{.}\PYG{n+na}{println}\PYG{o}{(}\PYG{n}{urlStringBuffer}\PYG{o}{.}\PYG{n+na}{toString}\PYG{o}{(}\PYG{o}{)}\PYG{o}{)}\PYG{o}{;}
              \PYG{k}{return} \PYG{n}{urlStringBuffer}\PYG{o}{.}\PYG{n+na}{toString}\PYG{o}{(}\PYG{o}{)}\PYG{o}{;}
          \PYG{o}{\PYGZcb{}}\PYG{o}{\PYGZcb{}}
\end{Verbatim}

Con el metodo getQueryBy formamos la consulta del servicio. Solamente hay que especificar el service\_name y todos los atributos correspondiente a ese servicio.


\subsection{Agregar un DTO}
\label{appAndroid:agregar-un-dto}
\begin{Verbatim}[commandchars=\\\{\}]
\PYG{n+nd}{@JsonSerialize}\PYG{o}{(}\PYG{n}{include} \PYG{o}{=} \PYG{n}{JsonSerialize}\PYG{o}{.}\PYG{n+na}{Inclusion}\PYG{o}{.}\PYG{n+na}{NON\PYGZus{}NULL}\PYG{o}{)}
\PYG{n+nd}{@JsonIgnoreProperties}\PYG{o}{(}\PYG{n}{ignoreUnknown} \PYG{o}{=} \PYG{k+kc}{true}\PYG{o}{)}

\PYG{k+kd}{public} \PYG{k+kd}{class} \PYG{n+nc}{SkillDTO} \PYG{k+kd}{extends} \PYG{n}{AbstractDTO} \PYG{o}{\PYGZob{}}
          \PYG{k+kd}{public} \PYG{n}{Skill} \PYG{n+nf}{getData}\PYG{o}{(}\PYG{o}{)} \PYG{o}{\PYGZob{}}
              \PYG{k}{return} \PYG{o}{(}\PYG{n}{Skill}\PYG{o}{)} \PYG{n}{data}\PYG{o}{;}
          \PYG{o}{\PYGZcb{}}

          \PYG{k+kd}{public} \PYG{k+kt}{void} \PYG{n+nf}{setData}\PYG{o}{(}\PYG{n}{Skill} \PYG{n}{data}\PYG{o}{)} \PYG{o}{\PYGZob{}}
              \PYG{k}{this}\PYG{o}{.}\PYG{n+na}{data} \PYG{o}{=} \PYG{n}{data}\PYG{o}{;}
          \PYG{o}{\PYGZcb{}}

          \PYG{k+kd}{public} \PYG{n}{String}\PYG{o}{[}\PYG{o}{]} \PYG{n+nf}{getErrors}\PYG{o}{(}\PYG{o}{)} \PYG{o}{\PYGZob{}}
              \PYG{k}{return} \PYG{o}{(}\PYG{n}{String}\PYG{o}{[}\PYG{o}{]}\PYG{o}{)} \PYG{n}{errors}\PYG{o}{;}
          \PYG{o}{\PYGZcb{}}

          \PYG{k+kd}{public} \PYG{k+kt}{void} \PYG{n+nf}{setErrors}\PYG{o}{(}\PYG{n}{String}\PYG{o}{[}\PYG{o}{]} \PYG{n}{errors}\PYG{o}{)} \PYG{o}{\PYGZob{}}
              \PYG{k}{this}\PYG{o}{.}\PYG{n+na}{errors} \PYG{o}{=} \PYG{n}{errors}\PYG{o}{;}
          \PYG{o}{\PYGZcb{}}
\PYG{o}{\PYGZcb{}}
\end{Verbatim}

Se debe definir los metodos \emph{getData} \emph{setData} \emph{getErrors} y \emph{setErrors}

Todo clase que hereda de AbstractServices contiene los metodos
\emph{getDataofDTO}, \emph{postDataofDTO}, \emph{deleteDataofDTO} y \emph{putDataOfDto}
que es donde se coloca la lógica que permite transfomar una query en un objeto que pertenece a una clase del dominio del problema.


\subsection{Definir la clase contenedora de la información}
\label{appAndroid:definir-la-clase-contenedora-de-la-informacion}
Se debe definir la clase que nos va a permitir manejar la información que nos va a devolver el getData del DTO.
Se le deben definir los atributos correspondientes y sus getters y setters.


\section{Contacto}
\label{appAndroid:contacto}
Por cualquier inquietud, duda, consulta, usted puede enviar un email a la casilla de correos \url{mailto:linkedunservices@gmail.com} y se le responderá a la brevedad.
También puede contribuir al proyecto, reportar bugs, etc en la página de github del proyecto.


\section{Referencias}
\label{appAndroid:referencias}\begin{itemize}
\item {} 
{[}1{]} Android Studio: \url{https://developer.android.com/studio/index.html?hl=es-419}

\item {} 
{[}2{]} Firebase: \url{https://firebase.google.com/?hl=es}

\item {} 
{[}3{]} Firebase Cloud Messages: \url{https://firebase.google.com/docs/cloud-messaging/?hl=es}

\item {} 
{[}4{]} Código del repositorio oficial de Linkedun: \url{https://github.com/tomasmussi/taller2}

\end{itemize}


\chapter{SharedServer}
\label{index:sharedserver}

\section{Objetivo}
\label{sharedServer:objetivo}\label{sharedServer::doc}
Documentar información técnica del \emph{Shared Server} para poder ejecutar, extender y mejorar dicho servidor.


\section{Requisitos:}
\label{sharedServer:requisitos}
El servidor de la aplicación \emph{Linkedun} al cuál se consultan todos los servicios está desarrollado en \emph{NodeJS}, utiliza los siguientes módulos:
\begin{itemize}
\item {} 
\textbf{pg:} V6.1.0 “PostgreSQL” {[}2{]}

\item {} 
\textbf{Express:} V4.13.3 “Framework de infraestructura web rápida, minimalista y flexible para las aplicaciones Node.js.”{[}3{]}

\item {} 
\textbf{Winston:} V\textasciicircum{}2.3.0 “Logger”{[}4{]}

\item {} 
\textbf{pg-promise:} V \textasciicircum{}3.2.3 “Promises interface for PostgreSQL”{[}5{]}

\item {} 
\textbf{bluebird:} V\textasciicircum{}3.3.4 “Promises in NodeJS”{[}6{]}

\item {} 
\textbf{body-parser:} V \textasciicircum{}1.15.2 “Node.js body parsing middleware”{[}7{]}

\item {} 
\textbf{ejs:} V2.4.1 “Embedded JavaScript templates”{[}8{]}

\end{itemize}


\section{Distribución}
\label{sharedServer:distribucion}
El código puede ser descargado de la página oficial de \emph{github {[}10{]}}.
Sistema operativo
El desarrollo de la aplicación se ha realizado bajo el sistema operativo Linux. De querer desarrollar en Windows o Mac, deberá considerar la portabilidad de todas las dependencias a esos sistemas operativos y cómo realizar la compilación del proyecto.


\section{Construcción de Imágenes Docker (Builds)}
\label{sharedServer:construccion-de-imagenes-docker-builds}
Este software está diseñado para ejecutarse en un contenedor Docker. Para ello, primero necesitamos construir la imagen de PostgreSQL para la base de datos que le da infraestructura a este servidor.

Posicionarse dentro de la carpeta /database/ dentro de la carpeta principal del proyecto descargado por git. Luego ejecutar el siguiente comando:
\begin{itemize}
\item {} 
docker build -t sharedserver-database .

\end{itemize}

Esto generará la dockerizacion de la base de datos, luego hay que posicionarse en la carpeta principal ( taller2/SharedServerHeroku/ ) y ejecutar el siguiente comando:
\begin{itemize}
\item {} 
docker-compose up

\end{itemize}

Este comando dockerizara NodeJs y unirá las dos aplicaciones (PostgreSQL y NodeJs) quedando el servidor operativo. Para detener la ejecución del servidor ejecutar Control + C.


\section{Ejecución}
\label{sharedServer:ejecucion}
A traves del comando:
\begin{itemize}
\item {} 
docker-compose up

\end{itemize}

Ejecutado en la carpeta principal, luego de haber realizado el instructivo de construcción, deberá dejar la aplicación de NodeJS escuchando en el puerto 5000 del servidor localhost. Asi como tambien la base de datos escuchando en el puerto 5432 del servidor localhost. Listas para ser utilizadas, exponiendo una web de administración descrita en el Manual del Administrador Shared.


\section{Estructura}
\label{sharedServer:estructura}
Las consultas a este servidor están expuestas en el index.js en el directorio raíz del proyecto y son realizadas a través la API {[}1{]}, resumidamente posee los siguientes servicios:

\begin{Verbatim}[commandchars=\\\{\}]
app.get(\PYGZsq{}/job\PYGZus{}positions\PYGZsq{},db.getAllJobPositions);
app.get(\PYGZsq{}/job\PYGZus{}positions/categories/:name\PYGZsq{},db.getAllJobPositionsByCategory);
app.get(\PYGZsq{}/categories\PYGZsq{},db.getAllCategories);
app.get(\PYGZsq{}/skills\PYGZsq{},db.getAllSkills);
app.post(\PYGZsq{}/skills/categories/:category\PYGZsq{},db.postskill);
app.post(\PYGZsq{}/job\PYGZus{}positions/categories/:category\PYGZsq{},db.postJobPosition);
app.post(\PYGZsq{}/categories/\PYGZsq{},db.postCategory);
app.put(\PYGZsq{}/categories/:name\PYGZsq{},db.putCategory);
app.put(\PYGZsq{}/job\PYGZus{}positions/categories/:category/:name\PYGZsq{},db.putJobPosition);
app.put(\PYGZsq{}/skills/categories/:category/:name\PYGZsq{},db.putSkill);
app.delete(\PYGZsq{}/job\PYGZus{}positions/categories/:category/:name\PYGZsq{},db.deleteJobPosition);
app.delete(\PYGZsq{}/skills/categories/:category/:name\PYGZsq{},db.deleteSkill);
app.delete(\PYGZsq{}/categories/:name\PYGZsq{},db.deleteCategory);
\end{Verbatim}

Los cuales se encuentran implementados bajo los métodos que expone el módulo \emph{queries.js}, donde se encuentran los accesos a la base de datos pertinentes en cada servicio. Todas las respuestas de dicho servicio son en formato JSON.
En cuanto a la web de administración esta se encuentra en el directorio \emph{view/pages/} donde estan los archivos principales de la web y los controladores separados por skills, jobPosition y category.


\section{Base de datos}
\label{sharedServer:base-de-datos}
Este servidor posee una base de datos \emph{PostgreSQL} donde se almacenan los \emph{jobPositions}, los \emph{skills} y las \emph{categories}. La estructura de la misma puede ser representada a través del siguiente diagrama:

\noindent\sphinxincludegraphics{{diagramaBaseDeDatos}.png}

Para una mejor comprensión puede recurrir al archivo sharedserver.sql el cual contendrá la creación de las 3 tablas (category, job\_position y skill) con sus respectivos atributos y claves.


\section{Administracion web}
\label{sharedServer:administracion-web}
El Shared Server posee una web de administración online en la plataforma Heroku {[}9{]}. Esta provee de las acciones denominadas alta, baja, modificación y borrado de jobPositions, skills y  categories donde se puede ver este tema más detallado en el Manual del Administrador Shared. Dicha web se encuentra en la url : \url{https://guarded-sands-84788.herokuapp.com/\#/}
Cabe destacar que este dominio en Heroku posee las mismas características que el dominio local en el cual se trato en este documento.
API
El Shared Server respeta la API-REST{[}1{]} para la interacción con el App Server y posteriormente la aplicación Android.


\section{Contacto}
\label{sharedServer:contacto}
Por cualquier inquietud, duda, consulta, usted puede enviar un email a la casilla de correos
\href{mailto:linkedunservices@gmail.com}{linkedunservices@gmail.com} y se le responderá a la brevedad.
También puede contribuir al proyecto, reportar bugs, etc en la página de github del proyecto.


\section{Referencias}
\label{sharedServer:referencias}\begin{itemize}
\item {} 
{[}1{]} API Restful Shared Server : \url{http://rebilly.github.io/ReDoc/?url=https://gist.githubusercontent.com/NickCis/d6a8132a228440c41889b4e0003efc3b/raw/jobify-shared-api.yaml}

\item {} 
{[}2{]}pg : \url{https://www.npmjs.com/package/pg}

\item {} 
{[}3{]}Express : \url{https://www.npmjs.com/package/express}

\item {} 
{[}4{]}Winston : \url{https://www.npmjs.com/package/winston}

\item {} 
{[}5{]}pg-promise: \url{https://www.npmjs.com/package/pg-promise}

\item {} 
{[}6{]}bluebird : \url{https://www.npmjs.com/package/bluebird}

\item {} 
{[}7{]}body-parser: \url{https://www.npmjs.com/package/body-parser}

\item {} 
{[}8{]}ejs : \url{https://www.npmjs.com/package/ejs}

\item {} 
{[}9{]}Heroku : \url{https://www.heroku.com/}

\end{itemize}



\renewcommand{\indexname}{Índice}
\printindex
\end{document}
