% Generated by Sphinx.
\def\sphinxdocclass{report}
\newif\ifsphinxKeepOldNames \sphinxKeepOldNamestrue
\documentclass[letterpaper,10pt,spanish]{sphinxmanual}
\usepackage{iftex}

\ifPDFTeX
  \usepackage[utf8]{inputenc}
\fi
\ifdefined\DeclareUnicodeCharacter
  \DeclareUnicodeCharacter{00A0}{\nobreakspace}
\fi
\usepackage{cmap}
\usepackage[T1]{fontenc}
\usepackage{amsmath,amssymb,amstext}
\usepackage{babel}
\usepackage{times}
\usepackage[Sonny]{fncychap}
\usepackage{longtable}
\usepackage{sphinx}
\usepackage{multirow}
\usepackage{eqparbox}

\addto\captionsspanish{\renewcommand{\contentsname}{Contents:}}

\addto\captionsspanish{\renewcommand{\figurename}{Figura }}
\addto\captionsspanish{\renewcommand{\tablename}{Tabla }}
\SetupFloatingEnvironment{literal-block}{name=Lista }

\addto\extrasspanish{\def\pageautorefname{página}}

\setcounter{tocdepth}{1}


\title{Jobify - Taller de Programación II Documentation}
\date{nov. 30, 2016}
\release{1.0}
\author{Luis Arancibia, Tomás Mussi, Alfredo, Ezequiel Dufau}
\newcommand{\sphinxlogo}{}
\renewcommand{\releasename}{Publicación}
\makeindex

\makeatletter
\def\PYG@reset{\let\PYG@it=\relax \let\PYG@bf=\relax%
    \let\PYG@ul=\relax \let\PYG@tc=\relax%
    \let\PYG@bc=\relax \let\PYG@ff=\relax}
\def\PYG@tok#1{\csname PYG@tok@#1\endcsname}
\def\PYG@toks#1+{\ifx\relax#1\empty\else%
    \PYG@tok{#1}\expandafter\PYG@toks\fi}
\def\PYG@do#1{\PYG@bc{\PYG@tc{\PYG@ul{%
    \PYG@it{\PYG@bf{\PYG@ff{#1}}}}}}}
\def\PYG#1#2{\PYG@reset\PYG@toks#1+\relax+\PYG@do{#2}}

\expandafter\def\csname PYG@tok@gd\endcsname{\def\PYG@tc##1{\textcolor[rgb]{0.63,0.00,0.00}{##1}}}
\expandafter\def\csname PYG@tok@gu\endcsname{\let\PYG@bf=\textbf\def\PYG@tc##1{\textcolor[rgb]{0.50,0.00,0.50}{##1}}}
\expandafter\def\csname PYG@tok@gt\endcsname{\def\PYG@tc##1{\textcolor[rgb]{0.00,0.27,0.87}{##1}}}
\expandafter\def\csname PYG@tok@gs\endcsname{\let\PYG@bf=\textbf}
\expandafter\def\csname PYG@tok@gr\endcsname{\def\PYG@tc##1{\textcolor[rgb]{1.00,0.00,0.00}{##1}}}
\expandafter\def\csname PYG@tok@cm\endcsname{\let\PYG@it=\textit\def\PYG@tc##1{\textcolor[rgb]{0.25,0.50,0.56}{##1}}}
\expandafter\def\csname PYG@tok@vg\endcsname{\def\PYG@tc##1{\textcolor[rgb]{0.73,0.38,0.84}{##1}}}
\expandafter\def\csname PYG@tok@vi\endcsname{\def\PYG@tc##1{\textcolor[rgb]{0.73,0.38,0.84}{##1}}}
\expandafter\def\csname PYG@tok@mh\endcsname{\def\PYG@tc##1{\textcolor[rgb]{0.13,0.50,0.31}{##1}}}
\expandafter\def\csname PYG@tok@cs\endcsname{\def\PYG@tc##1{\textcolor[rgb]{0.25,0.50,0.56}{##1}}\def\PYG@bc##1{\setlength{\fboxsep}{0pt}\colorbox[rgb]{1.00,0.94,0.94}{\strut ##1}}}
\expandafter\def\csname PYG@tok@ge\endcsname{\let\PYG@it=\textit}
\expandafter\def\csname PYG@tok@vc\endcsname{\def\PYG@tc##1{\textcolor[rgb]{0.73,0.38,0.84}{##1}}}
\expandafter\def\csname PYG@tok@il\endcsname{\def\PYG@tc##1{\textcolor[rgb]{0.13,0.50,0.31}{##1}}}
\expandafter\def\csname PYG@tok@go\endcsname{\def\PYG@tc##1{\textcolor[rgb]{0.20,0.20,0.20}{##1}}}
\expandafter\def\csname PYG@tok@cp\endcsname{\def\PYG@tc##1{\textcolor[rgb]{0.00,0.44,0.13}{##1}}}
\expandafter\def\csname PYG@tok@gi\endcsname{\def\PYG@tc##1{\textcolor[rgb]{0.00,0.63,0.00}{##1}}}
\expandafter\def\csname PYG@tok@gh\endcsname{\let\PYG@bf=\textbf\def\PYG@tc##1{\textcolor[rgb]{0.00,0.00,0.50}{##1}}}
\expandafter\def\csname PYG@tok@ni\endcsname{\let\PYG@bf=\textbf\def\PYG@tc##1{\textcolor[rgb]{0.84,0.33,0.22}{##1}}}
\expandafter\def\csname PYG@tok@nl\endcsname{\let\PYG@bf=\textbf\def\PYG@tc##1{\textcolor[rgb]{0.00,0.13,0.44}{##1}}}
\expandafter\def\csname PYG@tok@nn\endcsname{\let\PYG@bf=\textbf\def\PYG@tc##1{\textcolor[rgb]{0.05,0.52,0.71}{##1}}}
\expandafter\def\csname PYG@tok@no\endcsname{\def\PYG@tc##1{\textcolor[rgb]{0.38,0.68,0.84}{##1}}}
\expandafter\def\csname PYG@tok@na\endcsname{\def\PYG@tc##1{\textcolor[rgb]{0.25,0.44,0.63}{##1}}}
\expandafter\def\csname PYG@tok@nb\endcsname{\def\PYG@tc##1{\textcolor[rgb]{0.00,0.44,0.13}{##1}}}
\expandafter\def\csname PYG@tok@nc\endcsname{\let\PYG@bf=\textbf\def\PYG@tc##1{\textcolor[rgb]{0.05,0.52,0.71}{##1}}}
\expandafter\def\csname PYG@tok@nd\endcsname{\let\PYG@bf=\textbf\def\PYG@tc##1{\textcolor[rgb]{0.33,0.33,0.33}{##1}}}
\expandafter\def\csname PYG@tok@ne\endcsname{\def\PYG@tc##1{\textcolor[rgb]{0.00,0.44,0.13}{##1}}}
\expandafter\def\csname PYG@tok@nf\endcsname{\def\PYG@tc##1{\textcolor[rgb]{0.02,0.16,0.49}{##1}}}
\expandafter\def\csname PYG@tok@si\endcsname{\let\PYG@it=\textit\def\PYG@tc##1{\textcolor[rgb]{0.44,0.63,0.82}{##1}}}
\expandafter\def\csname PYG@tok@s2\endcsname{\def\PYG@tc##1{\textcolor[rgb]{0.25,0.44,0.63}{##1}}}
\expandafter\def\csname PYG@tok@nt\endcsname{\let\PYG@bf=\textbf\def\PYG@tc##1{\textcolor[rgb]{0.02,0.16,0.45}{##1}}}
\expandafter\def\csname PYG@tok@nv\endcsname{\def\PYG@tc##1{\textcolor[rgb]{0.73,0.38,0.84}{##1}}}
\expandafter\def\csname PYG@tok@s1\endcsname{\def\PYG@tc##1{\textcolor[rgb]{0.25,0.44,0.63}{##1}}}
\expandafter\def\csname PYG@tok@ch\endcsname{\let\PYG@it=\textit\def\PYG@tc##1{\textcolor[rgb]{0.25,0.50,0.56}{##1}}}
\expandafter\def\csname PYG@tok@m\endcsname{\def\PYG@tc##1{\textcolor[rgb]{0.13,0.50,0.31}{##1}}}
\expandafter\def\csname PYG@tok@gp\endcsname{\let\PYG@bf=\textbf\def\PYG@tc##1{\textcolor[rgb]{0.78,0.36,0.04}{##1}}}
\expandafter\def\csname PYG@tok@sh\endcsname{\def\PYG@tc##1{\textcolor[rgb]{0.25,0.44,0.63}{##1}}}
\expandafter\def\csname PYG@tok@ow\endcsname{\let\PYG@bf=\textbf\def\PYG@tc##1{\textcolor[rgb]{0.00,0.44,0.13}{##1}}}
\expandafter\def\csname PYG@tok@sx\endcsname{\def\PYG@tc##1{\textcolor[rgb]{0.78,0.36,0.04}{##1}}}
\expandafter\def\csname PYG@tok@bp\endcsname{\def\PYG@tc##1{\textcolor[rgb]{0.00,0.44,0.13}{##1}}}
\expandafter\def\csname PYG@tok@c1\endcsname{\let\PYG@it=\textit\def\PYG@tc##1{\textcolor[rgb]{0.25,0.50,0.56}{##1}}}
\expandafter\def\csname PYG@tok@o\endcsname{\def\PYG@tc##1{\textcolor[rgb]{0.40,0.40,0.40}{##1}}}
\expandafter\def\csname PYG@tok@kc\endcsname{\let\PYG@bf=\textbf\def\PYG@tc##1{\textcolor[rgb]{0.00,0.44,0.13}{##1}}}
\expandafter\def\csname PYG@tok@c\endcsname{\let\PYG@it=\textit\def\PYG@tc##1{\textcolor[rgb]{0.25,0.50,0.56}{##1}}}
\expandafter\def\csname PYG@tok@mf\endcsname{\def\PYG@tc##1{\textcolor[rgb]{0.13,0.50,0.31}{##1}}}
\expandafter\def\csname PYG@tok@err\endcsname{\def\PYG@bc##1{\setlength{\fboxsep}{0pt}\fcolorbox[rgb]{1.00,0.00,0.00}{1,1,1}{\strut ##1}}}
\expandafter\def\csname PYG@tok@mb\endcsname{\def\PYG@tc##1{\textcolor[rgb]{0.13,0.50,0.31}{##1}}}
\expandafter\def\csname PYG@tok@ss\endcsname{\def\PYG@tc##1{\textcolor[rgb]{0.32,0.47,0.09}{##1}}}
\expandafter\def\csname PYG@tok@sr\endcsname{\def\PYG@tc##1{\textcolor[rgb]{0.14,0.33,0.53}{##1}}}
\expandafter\def\csname PYG@tok@mo\endcsname{\def\PYG@tc##1{\textcolor[rgb]{0.13,0.50,0.31}{##1}}}
\expandafter\def\csname PYG@tok@kd\endcsname{\let\PYG@bf=\textbf\def\PYG@tc##1{\textcolor[rgb]{0.00,0.44,0.13}{##1}}}
\expandafter\def\csname PYG@tok@mi\endcsname{\def\PYG@tc##1{\textcolor[rgb]{0.13,0.50,0.31}{##1}}}
\expandafter\def\csname PYG@tok@kn\endcsname{\let\PYG@bf=\textbf\def\PYG@tc##1{\textcolor[rgb]{0.00,0.44,0.13}{##1}}}
\expandafter\def\csname PYG@tok@cpf\endcsname{\let\PYG@it=\textit\def\PYG@tc##1{\textcolor[rgb]{0.25,0.50,0.56}{##1}}}
\expandafter\def\csname PYG@tok@kr\endcsname{\let\PYG@bf=\textbf\def\PYG@tc##1{\textcolor[rgb]{0.00,0.44,0.13}{##1}}}
\expandafter\def\csname PYG@tok@s\endcsname{\def\PYG@tc##1{\textcolor[rgb]{0.25,0.44,0.63}{##1}}}
\expandafter\def\csname PYG@tok@kp\endcsname{\def\PYG@tc##1{\textcolor[rgb]{0.00,0.44,0.13}{##1}}}
\expandafter\def\csname PYG@tok@w\endcsname{\def\PYG@tc##1{\textcolor[rgb]{0.73,0.73,0.73}{##1}}}
\expandafter\def\csname PYG@tok@kt\endcsname{\def\PYG@tc##1{\textcolor[rgb]{0.56,0.13,0.00}{##1}}}
\expandafter\def\csname PYG@tok@sc\endcsname{\def\PYG@tc##1{\textcolor[rgb]{0.25,0.44,0.63}{##1}}}
\expandafter\def\csname PYG@tok@sb\endcsname{\def\PYG@tc##1{\textcolor[rgb]{0.25,0.44,0.63}{##1}}}
\expandafter\def\csname PYG@tok@k\endcsname{\let\PYG@bf=\textbf\def\PYG@tc##1{\textcolor[rgb]{0.00,0.44,0.13}{##1}}}
\expandafter\def\csname PYG@tok@se\endcsname{\let\PYG@bf=\textbf\def\PYG@tc##1{\textcolor[rgb]{0.25,0.44,0.63}{##1}}}
\expandafter\def\csname PYG@tok@sd\endcsname{\let\PYG@it=\textit\def\PYG@tc##1{\textcolor[rgb]{0.25,0.44,0.63}{##1}}}

\def\PYGZbs{\char`\\}
\def\PYGZus{\char`\_}
\def\PYGZob{\char`\{}
\def\PYGZcb{\char`\}}
\def\PYGZca{\char`\^}
\def\PYGZam{\char`\&}
\def\PYGZlt{\char`\<}
\def\PYGZgt{\char`\>}
\def\PYGZsh{\char`\#}
\def\PYGZpc{\char`\%}
\def\PYGZdl{\char`\$}
\def\PYGZhy{\char`\-}
\def\PYGZsq{\char`\'}
\def\PYGZdq{\char`\"}
\def\PYGZti{\char`\~}
% for compatibility with earlier versions
\def\PYGZat{@}
\def\PYGZlb{[}
\def\PYGZrb{]}
\makeatother

\renewcommand\PYGZsq{\textquotesingle}

\begin{document}
\shorthandoff{"}
\maketitle
\tableofcontents
\phantomsection\label{index::doc}



\chapter{Introducción}
\label{introduccion:welcome-to-jobify-taller-de-programacion-ii-s-documentation}\label{introduccion::doc}\label{introduccion:introduccion}
El trabajo práctico consta de 4 aplicaciones que funcionan para dar servicio a Jobify, una red social de profesionales.
\begin{itemize}
\item {} 
\textbf{AppAndroid}: aplicación android que permite a un usuario registrarse, editar su perfil en la red social, chatear, votar, encontrar profesionales cercanos y agregar contactos a la lista de amigos. Desarrollado en Android

\item {} 
\textbf{AppServer}: servidor responsable de toda la lógica de Jobify, da servicios a la aplicación android. Además persiste todos los datos del usuario. Desarrollado en c++ y levelDB como base de datos.

\item {} 
\textbf{SharedServer}: servidor que contiene datos de uso común de la aplicación. Utilizando Heroku para hacer un deploy de la aplicación. Desarrollado en node.js y postgreSQL como base de datos

\item {} 
\textbf{WebAdmin}: web que permite editar los datos utilizado por el SharedServer.

\end{itemize}


\chapter{Instalación}
\label{instalacion:instalacion}\label{instalacion::doc}
Contenido:


\section{Heroku-test}
\label{Heroku-test::doc}\label{Heroku-test:heroku-test}\begin{itemize}
\item {} 
Instalar python
\begin{quote}

sudo apt-get install python
\end{quote}

\item {} 
Instalar pip
\begin{quote}

sudo easy\_install pip
\begin{quote}

Aclaración: si no funciona instalar sudo apt-get install python-setuptools
\end{quote}
\end{quote}

\item {} 
Instalar nose
\begin{quote}

pip install nose

sudo pip install requests
sudo pip install -U mock
\end{quote}

\end{itemize}

\textbf{Prueba de test:}

Primero asegurarse de que este corriendo el servidor localmente, si no es asi ejecutar
\begin{quote}

heroku local web
\end{quote}

Luego ejecutar los test con el siguiente comando
\begin{quote}

heroku local test
\end{quote}


\section{Coverage}
\label{Coverage::doc}\label{Coverage:coverage}
\#Intalar lcov

sudo apt-get install lcov

\#Uso
Hay que compilar el server como siempre y una vez finalizado, ejecutar el script ./coverage.sh en la carpeta server. Se genera la documentación y automaticamente se abre el firefox con un informe de coverage.
Se probó con los 17 test que están implementados.


\section{Curl-Curlpp}
\label{Curl-Curlpp:curl-curlpp}\label{Curl-Curlpp::doc}
Bajar paquete
Para ubuntu 14.04 el paquete es: libcurl4-openssl-dev

sudo apt-get install libcurl4-openssl-dev libboost-all-dev

Descargar y destarear curl

wget \url{https://storage.googleapis.com/google-code-archive-downloads/v2/code.google.com/curlpp/curlpp-0.7.3.tar.gz}
tar zxf curlpp-0.7.3.tar.gz

Instalar

cd curlpp-0.7.3
./configure
make
sudo make install


\section{Gtest}
\label{Gtest:gtest}\label{Gtest::doc}
\textbf{1. Descargar el repositorio}
git clone \url{https://github.com/google/googletest.git}

\textbf{2. Compilar la libreria}

cd googletest \textless{}br\textgreater{}

cmake -DBUILD\_SHARED\_LIBS=ON . \textless{}br\textgreater{}
make \textless{}br\textgreater{}
cd googletest \textless{}br\textgreater{}

sudo cp -a include/gtest /usr/include \textless{}br\textgreater{}

cmake -DBUILD\_SHARED\_LIBS=ON . \textless{}br\textgreater{}
make \textless{}br\textgreater{}
sudo cp -a libgtest\_main.so libgtest.so /usr/lib/ \textless{}br\textgreater{}

\textbf{3. Updatear la cache del linker} \textless{}br\textgreater{}
\$ sudo ldconfig -v \textbar{} grep gtest \textless{}br\textgreater{}

Deberiamos ver esto \textless{}br\textgreater{}
libgtest.so.0 -\textgreater{} libgtest.so.0.0.0 \textless{}br\textgreater{}
libgtest\_main.so.0 -\textgreater{} libgtest\_main.so.0.0.0 \textless{}br\textgreater{}


\section{Heroku-Node-Postgresql}
\label{Heroku-Node-Postgresql::doc}\label{Heroku-Node-Postgresql:heroku-node-postgresql}
\#\# 1- Descargar el proyecto del GitHub:

git clone -b {[}branch{]} {[}remote\_repo{]} en nuestro caso seria:
\begin{quote}

git clone -b sharedServer \url{https://github.com/tomasmussi/taller2.git}
\end{quote}

\#\# 2- Instalar Node.js:
\begin{quote}

curl -sL \url{https://deb.nodesource.com/setup\_4.x} \textbar{} sudo -E bash -

sudo apt-get install -y nodejs
\end{quote}

Ejecutar:
\begin{quote}

node -version
\end{quote}

y verificar que sea V4.5.0

\#\# 3- Instalar Heroku:

Chequear que ruby este instalado con:
\begin{quote}

ruby -version
\end{quote}

y si esta instalado ejecutar:
\begin{quote}

wget -O- \url{https://toolbelt.heroku.com/install-ubuntu.sh} \textbar{} sh
\end{quote}

\#\# 4- Instalar Postgresql:
\begin{quote}

sudo apt-get install postgresql
\end{quote}

para chequear:
\begin{quote}

sudo su postgres
\end{quote}

cambio de usuario -\textgreater{} \href{mailto:postgres@alfredo-GFAST}{postgres@alfredo-GFAST}:/home/alfredo/Jobify/taller2/SharedServerHeroku\$

ejecutar:
\begin{quote}

exit
\end{quote}

para volver al usuario -\textgreater{} \href{mailto:alfredo@alfredo-GFAST}{alfredo@alfredo-GFAST}:/home/alfredo/Jobify/taller2/SharedServerHeroku\$

\#\# 5- Ingresar a la carpeta taller2/SharedServer y ejecutar:
\begin{quote}

heroku local database
\end{quote}

\#\# 6- Luego ejecutar:
\begin{quote}

heroku local web
\end{quote}

\#\# 7- Ingresar a:
\begin{quote}

\url{http://localhost:5000/}

\url{http://localhost:5000/job\_positions}
\end{quote}


\section{Jsoncpp-Mongoose-LevelDB}
\label{Jsoncpp-Mongoose-LevelDB:jsoncpp-mongoose-leveldb}\label{Jsoncpp-Mongoose-LevelDB::doc}
\#\# \textbf{1- Guia de instalacion leveldb}

Instalar git-core and libsnappy-dev \textless{}br\textgreater{}
\begin{quote}

sudo apt-get install git-core libsnappy-dev
\end{quote}

Clonar leveldb del repositorio de git \textless{}br\textgreater{}
\begin{quote}

git clone  \url{https://github.com/google/leveldb.git}
\end{quote}

Compilación \textless{}br\textgreater{}
\begin{quote}

cd leveldb
make
\end{quote}

Instalación \textless{}br\textgreater{}
\begin{quote}

cd out-shared
sudo cp --preserve=links libleveldb.* /usr/local/lib
cd ../include
sudo cp -R leveldb /usr/local/include/
sudo ldconfig
\end{quote}

\#\# \textbf{2- Instalacion mongoose-cpp y jsoncpp}

Para instalar mongoose-cpp y jsoncpp y poder compilar el server, se deben seguir estos pasos
\begin{enumerate}
\item {} 
Hacer un checkout del proyecto {[}jsoncpp{]}(\url{https://github.com/open-source-parsers/jsoncpp}) \textless{}br\textgreater{}
\begin{quote}

git clone \url{https://github.com/open-source-parsers/jsoncpp.git}
\end{quote}

\item {} \begin{description}
\item[{dentro de jsoncpp ejecutar comandos:}] \leavevmode
mkdir build
cd build
cmake ..
sudo make install

\end{description}

\end{enumerate}

Con esto logramos instalar jsoncpp en linux, creo que lo que hace es copiar todo el codigo a /usr/local/include y cuando compilas, se buscan todos los includes en \$PATH que contiene a /usr/local/include.
Y listo! Porque la carpeta de mongoose-cpp la tenemos dentro de nuestro proyecto, asi que los pasos son:
\begin{quote}

cd server
mkdir build \&\& cd build
cmake ..
make
./src/server
\end{quote}

Y ya está el server andando escuchando en localhost, puerto 8080. Abrir en el navegador:
\begin{quote}

localhost:8080/
\end{quote}


\section{Log4cpp}
\label{Log4cpp::doc}\label{Log4cpp:log4cpp}
Bajar ultima version de: \url{http://log4cpp.sourceforge.net/}

sudo ./configure

sudo make

sudo make check

sudo make install

sudo ldconfig



\renewcommand{\indexname}{Índice}
\printindex
\end{document}
